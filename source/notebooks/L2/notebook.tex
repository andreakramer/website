
% Default to the notebook output style

    


% Inherit from the specified cell style.




    
\documentclass[11pt]{article}

    
    
    \usepackage[T1]{fontenc}
    % Nicer default font (+ math font) than Computer Modern for most use cases
    \usepackage{mathpazo}

    % Basic figure setup, for now with no caption control since it's done
    % automatically by Pandoc (which extracts ![](path) syntax from Markdown).
    \usepackage{graphicx}
    % We will generate all images so they have a width \maxwidth. This means
    % that they will get their normal width if they fit onto the page, but
    % are scaled down if they would overflow the margins.
    \makeatletter
    \def\maxwidth{\ifdim\Gin@nat@width>\linewidth\linewidth
    \else\Gin@nat@width\fi}
    \makeatother
    \let\Oldincludegraphics\includegraphics
    % Set max figure width to be 80% of text width, for now hardcoded.
    \renewcommand{\includegraphics}[1]{\Oldincludegraphics[width=.8\maxwidth]{#1}}
    % Ensure that by default, figures have no caption (until we provide a
    % proper Figure object with a Caption API and a way to capture that
    % in the conversion process - todo).
    \usepackage{caption}
    \DeclareCaptionLabelFormat{nolabel}{}
    \captionsetup{labelformat=nolabel}

    \usepackage{adjustbox} % Used to constrain images to a maximum size 
    \usepackage{xcolor} % Allow colors to be defined
    \usepackage{enumerate} % Needed for markdown enumerations to work
    \usepackage{geometry} % Used to adjust the document margins
    \usepackage{amsmath} % Equations
    \usepackage{amssymb} % Equations
    \usepackage{textcomp} % defines textquotesingle
    % Hack from http://tex.stackexchange.com/a/47451/13684:
    \AtBeginDocument{%
        \def\PYZsq{\textquotesingle}% Upright quotes in Pygmentized code
    }
    \usepackage{upquote} % Upright quotes for verbatim code
    \usepackage{eurosym} % defines \euro
    \usepackage[mathletters]{ucs} % Extended unicode (utf-8) support
    \usepackage[utf8x]{inputenc} % Allow utf-8 characters in the tex document
    \usepackage{fancyvrb} % verbatim replacement that allows latex
    \usepackage{grffile} % extends the file name processing of package graphics 
                         % to support a larger range 
    % The hyperref package gives us a pdf with properly built
    % internal navigation ('pdf bookmarks' for the table of contents,
    % internal cross-reference links, web links for URLs, etc.)
    \usepackage{hyperref}
    \usepackage{longtable} % longtable support required by pandoc >1.10
    \usepackage{booktabs}  % table support for pandoc > 1.12.2
    \usepackage[inline]{enumitem} % IRkernel/repr support (it uses the enumerate* environment)
    \usepackage[normalem]{ulem} % ulem is needed to support strikethroughs (\sout)
                                % normalem makes italics be italics, not underlines
    

    
    
    % Colors for the hyperref package
    \definecolor{urlcolor}{rgb}{0,.145,.698}
    \definecolor{linkcolor}{rgb}{.71,0.21,0.01}
    \definecolor{citecolor}{rgb}{.12,.54,.11}

    % ANSI colors
    \definecolor{ansi-black}{HTML}{3E424D}
    \definecolor{ansi-black-intense}{HTML}{282C36}
    \definecolor{ansi-red}{HTML}{E75C58}
    \definecolor{ansi-red-intense}{HTML}{B22B31}
    \definecolor{ansi-green}{HTML}{00A250}
    \definecolor{ansi-green-intense}{HTML}{007427}
    \definecolor{ansi-yellow}{HTML}{DDB62B}
    \definecolor{ansi-yellow-intense}{HTML}{B27D12}
    \definecolor{ansi-blue}{HTML}{208FFB}
    \definecolor{ansi-blue-intense}{HTML}{0065CA}
    \definecolor{ansi-magenta}{HTML}{D160C4}
    \definecolor{ansi-magenta-intense}{HTML}{A03196}
    \definecolor{ansi-cyan}{HTML}{60C6C8}
    \definecolor{ansi-cyan-intense}{HTML}{258F8F}
    \definecolor{ansi-white}{HTML}{C5C1B4}
    \definecolor{ansi-white-intense}{HTML}{A1A6B2}

    % commands and environments needed by pandoc snippets
    % extracted from the output of `pandoc -s`
    \providecommand{\tightlist}{%
      \setlength{\itemsep}{0pt}\setlength{\parskip}{0pt}}
    \DefineVerbatimEnvironment{Highlighting}{Verbatim}{commandchars=\\\{\}}
    % Add ',fontsize=\small' for more characters per line
    \newenvironment{Shaded}{}{}
    \newcommand{\KeywordTok}[1]{\textcolor[rgb]{0.00,0.44,0.13}{\textbf{{#1}}}}
    \newcommand{\DataTypeTok}[1]{\textcolor[rgb]{0.56,0.13,0.00}{{#1}}}
    \newcommand{\DecValTok}[1]{\textcolor[rgb]{0.25,0.63,0.44}{{#1}}}
    \newcommand{\BaseNTok}[1]{\textcolor[rgb]{0.25,0.63,0.44}{{#1}}}
    \newcommand{\FloatTok}[1]{\textcolor[rgb]{0.25,0.63,0.44}{{#1}}}
    \newcommand{\CharTok}[1]{\textcolor[rgb]{0.25,0.44,0.63}{{#1}}}
    \newcommand{\StringTok}[1]{\textcolor[rgb]{0.25,0.44,0.63}{{#1}}}
    \newcommand{\CommentTok}[1]{\textcolor[rgb]{0.38,0.63,0.69}{\textit{{#1}}}}
    \newcommand{\OtherTok}[1]{\textcolor[rgb]{0.00,0.44,0.13}{{#1}}}
    \newcommand{\AlertTok}[1]{\textcolor[rgb]{1.00,0.00,0.00}{\textbf{{#1}}}}
    \newcommand{\FunctionTok}[1]{\textcolor[rgb]{0.02,0.16,0.49}{{#1}}}
    \newcommand{\RegionMarkerTok}[1]{{#1}}
    \newcommand{\ErrorTok}[1]{\textcolor[rgb]{1.00,0.00,0.00}{\textbf{{#1}}}}
    \newcommand{\NormalTok}[1]{{#1}}
    
    % Additional commands for more recent versions of Pandoc
    \newcommand{\ConstantTok}[1]{\textcolor[rgb]{0.53,0.00,0.00}{{#1}}}
    \newcommand{\SpecialCharTok}[1]{\textcolor[rgb]{0.25,0.44,0.63}{{#1}}}
    \newcommand{\VerbatimStringTok}[1]{\textcolor[rgb]{0.25,0.44,0.63}{{#1}}}
    \newcommand{\SpecialStringTok}[1]{\textcolor[rgb]{0.73,0.40,0.53}{{#1}}}
    \newcommand{\ImportTok}[1]{{#1}}
    \newcommand{\DocumentationTok}[1]{\textcolor[rgb]{0.73,0.13,0.13}{\textit{{#1}}}}
    \newcommand{\AnnotationTok}[1]{\textcolor[rgb]{0.38,0.63,0.69}{\textbf{\textit{{#1}}}}}
    \newcommand{\CommentVarTok}[1]{\textcolor[rgb]{0.38,0.63,0.69}{\textbf{\textit{{#1}}}}}
    \newcommand{\VariableTok}[1]{\textcolor[rgb]{0.10,0.09,0.49}{{#1}}}
    \newcommand{\ControlFlowTok}[1]{\textcolor[rgb]{0.00,0.44,0.13}{\textbf{{#1}}}}
    \newcommand{\OperatorTok}[1]{\textcolor[rgb]{0.40,0.40,0.40}{{#1}}}
    \newcommand{\BuiltInTok}[1]{{#1}}
    \newcommand{\ExtensionTok}[1]{{#1}}
    \newcommand{\PreprocessorTok}[1]{\textcolor[rgb]{0.74,0.48,0.00}{{#1}}}
    \newcommand{\AttributeTok}[1]{\textcolor[rgb]{0.49,0.56,0.16}{{#1}}}
    \newcommand{\InformationTok}[1]{\textcolor[rgb]{0.38,0.63,0.69}{\textbf{\textit{{#1}}}}}
    \newcommand{\WarningTok}[1]{\textcolor[rgb]{0.38,0.63,0.69}{\textbf{\textit{{#1}}}}}
    
    
    % Define a nice break command that doesn't care if a line doesn't already
    % exist.
    \def\br{\hspace*{\fill} \\* }
    % Math Jax compatability definitions
    \def\gt{>}
    \def\lt{<}
    % Document parameters
    \title{Lecture 2}
    
    
    

    % Pygments definitions
    
\makeatletter
\def\PY@reset{\let\PY@it=\relax \let\PY@bf=\relax%
    \let\PY@ul=\relax \let\PY@tc=\relax%
    \let\PY@bc=\relax \let\PY@ff=\relax}
\def\PY@tok#1{\csname PY@tok@#1\endcsname}
\def\PY@toks#1+{\ifx\relax#1\empty\else%
    \PY@tok{#1}\expandafter\PY@toks\fi}
\def\PY@do#1{\PY@bc{\PY@tc{\PY@ul{%
    \PY@it{\PY@bf{\PY@ff{#1}}}}}}}
\def\PY#1#2{\PY@reset\PY@toks#1+\relax+\PY@do{#2}}

\expandafter\def\csname PY@tok@w\endcsname{\def\PY@tc##1{\textcolor[rgb]{0.73,0.73,0.73}{##1}}}
\expandafter\def\csname PY@tok@c\endcsname{\let\PY@it=\textit\def\PY@tc##1{\textcolor[rgb]{0.25,0.50,0.50}{##1}}}
\expandafter\def\csname PY@tok@cp\endcsname{\def\PY@tc##1{\textcolor[rgb]{0.74,0.48,0.00}{##1}}}
\expandafter\def\csname PY@tok@k\endcsname{\let\PY@bf=\textbf\def\PY@tc##1{\textcolor[rgb]{0.00,0.50,0.00}{##1}}}
\expandafter\def\csname PY@tok@kp\endcsname{\def\PY@tc##1{\textcolor[rgb]{0.00,0.50,0.00}{##1}}}
\expandafter\def\csname PY@tok@kt\endcsname{\def\PY@tc##1{\textcolor[rgb]{0.69,0.00,0.25}{##1}}}
\expandafter\def\csname PY@tok@o\endcsname{\def\PY@tc##1{\textcolor[rgb]{0.40,0.40,0.40}{##1}}}
\expandafter\def\csname PY@tok@ow\endcsname{\let\PY@bf=\textbf\def\PY@tc##1{\textcolor[rgb]{0.67,0.13,1.00}{##1}}}
\expandafter\def\csname PY@tok@nb\endcsname{\def\PY@tc##1{\textcolor[rgb]{0.00,0.50,0.00}{##1}}}
\expandafter\def\csname PY@tok@nf\endcsname{\def\PY@tc##1{\textcolor[rgb]{0.00,0.00,1.00}{##1}}}
\expandafter\def\csname PY@tok@nc\endcsname{\let\PY@bf=\textbf\def\PY@tc##1{\textcolor[rgb]{0.00,0.00,1.00}{##1}}}
\expandafter\def\csname PY@tok@nn\endcsname{\let\PY@bf=\textbf\def\PY@tc##1{\textcolor[rgb]{0.00,0.00,1.00}{##1}}}
\expandafter\def\csname PY@tok@ne\endcsname{\let\PY@bf=\textbf\def\PY@tc##1{\textcolor[rgb]{0.82,0.25,0.23}{##1}}}
\expandafter\def\csname PY@tok@nv\endcsname{\def\PY@tc##1{\textcolor[rgb]{0.10,0.09,0.49}{##1}}}
\expandafter\def\csname PY@tok@no\endcsname{\def\PY@tc##1{\textcolor[rgb]{0.53,0.00,0.00}{##1}}}
\expandafter\def\csname PY@tok@nl\endcsname{\def\PY@tc##1{\textcolor[rgb]{0.63,0.63,0.00}{##1}}}
\expandafter\def\csname PY@tok@ni\endcsname{\let\PY@bf=\textbf\def\PY@tc##1{\textcolor[rgb]{0.60,0.60,0.60}{##1}}}
\expandafter\def\csname PY@tok@na\endcsname{\def\PY@tc##1{\textcolor[rgb]{0.49,0.56,0.16}{##1}}}
\expandafter\def\csname PY@tok@nt\endcsname{\let\PY@bf=\textbf\def\PY@tc##1{\textcolor[rgb]{0.00,0.50,0.00}{##1}}}
\expandafter\def\csname PY@tok@nd\endcsname{\def\PY@tc##1{\textcolor[rgb]{0.67,0.13,1.00}{##1}}}
\expandafter\def\csname PY@tok@s\endcsname{\def\PY@tc##1{\textcolor[rgb]{0.73,0.13,0.13}{##1}}}
\expandafter\def\csname PY@tok@sd\endcsname{\let\PY@it=\textit\def\PY@tc##1{\textcolor[rgb]{0.73,0.13,0.13}{##1}}}
\expandafter\def\csname PY@tok@si\endcsname{\let\PY@bf=\textbf\def\PY@tc##1{\textcolor[rgb]{0.73,0.40,0.53}{##1}}}
\expandafter\def\csname PY@tok@se\endcsname{\let\PY@bf=\textbf\def\PY@tc##1{\textcolor[rgb]{0.73,0.40,0.13}{##1}}}
\expandafter\def\csname PY@tok@sr\endcsname{\def\PY@tc##1{\textcolor[rgb]{0.73,0.40,0.53}{##1}}}
\expandafter\def\csname PY@tok@ss\endcsname{\def\PY@tc##1{\textcolor[rgb]{0.10,0.09,0.49}{##1}}}
\expandafter\def\csname PY@tok@sx\endcsname{\def\PY@tc##1{\textcolor[rgb]{0.00,0.50,0.00}{##1}}}
\expandafter\def\csname PY@tok@m\endcsname{\def\PY@tc##1{\textcolor[rgb]{0.40,0.40,0.40}{##1}}}
\expandafter\def\csname PY@tok@gh\endcsname{\let\PY@bf=\textbf\def\PY@tc##1{\textcolor[rgb]{0.00,0.00,0.50}{##1}}}
\expandafter\def\csname PY@tok@gu\endcsname{\let\PY@bf=\textbf\def\PY@tc##1{\textcolor[rgb]{0.50,0.00,0.50}{##1}}}
\expandafter\def\csname PY@tok@gd\endcsname{\def\PY@tc##1{\textcolor[rgb]{0.63,0.00,0.00}{##1}}}
\expandafter\def\csname PY@tok@gi\endcsname{\def\PY@tc##1{\textcolor[rgb]{0.00,0.63,0.00}{##1}}}
\expandafter\def\csname PY@tok@gr\endcsname{\def\PY@tc##1{\textcolor[rgb]{1.00,0.00,0.00}{##1}}}
\expandafter\def\csname PY@tok@ge\endcsname{\let\PY@it=\textit}
\expandafter\def\csname PY@tok@gs\endcsname{\let\PY@bf=\textbf}
\expandafter\def\csname PY@tok@gp\endcsname{\let\PY@bf=\textbf\def\PY@tc##1{\textcolor[rgb]{0.00,0.00,0.50}{##1}}}
\expandafter\def\csname PY@tok@go\endcsname{\def\PY@tc##1{\textcolor[rgb]{0.53,0.53,0.53}{##1}}}
\expandafter\def\csname PY@tok@gt\endcsname{\def\PY@tc##1{\textcolor[rgb]{0.00,0.27,0.87}{##1}}}
\expandafter\def\csname PY@tok@err\endcsname{\def\PY@bc##1{\setlength{\fboxsep}{0pt}\fcolorbox[rgb]{1.00,0.00,0.00}{1,1,1}{\strut ##1}}}
\expandafter\def\csname PY@tok@kc\endcsname{\let\PY@bf=\textbf\def\PY@tc##1{\textcolor[rgb]{0.00,0.50,0.00}{##1}}}
\expandafter\def\csname PY@tok@kd\endcsname{\let\PY@bf=\textbf\def\PY@tc##1{\textcolor[rgb]{0.00,0.50,0.00}{##1}}}
\expandafter\def\csname PY@tok@kn\endcsname{\let\PY@bf=\textbf\def\PY@tc##1{\textcolor[rgb]{0.00,0.50,0.00}{##1}}}
\expandafter\def\csname PY@tok@kr\endcsname{\let\PY@bf=\textbf\def\PY@tc##1{\textcolor[rgb]{0.00,0.50,0.00}{##1}}}
\expandafter\def\csname PY@tok@bp\endcsname{\def\PY@tc##1{\textcolor[rgb]{0.00,0.50,0.00}{##1}}}
\expandafter\def\csname PY@tok@fm\endcsname{\def\PY@tc##1{\textcolor[rgb]{0.00,0.00,1.00}{##1}}}
\expandafter\def\csname PY@tok@vc\endcsname{\def\PY@tc##1{\textcolor[rgb]{0.10,0.09,0.49}{##1}}}
\expandafter\def\csname PY@tok@vg\endcsname{\def\PY@tc##1{\textcolor[rgb]{0.10,0.09,0.49}{##1}}}
\expandafter\def\csname PY@tok@vi\endcsname{\def\PY@tc##1{\textcolor[rgb]{0.10,0.09,0.49}{##1}}}
\expandafter\def\csname PY@tok@vm\endcsname{\def\PY@tc##1{\textcolor[rgb]{0.10,0.09,0.49}{##1}}}
\expandafter\def\csname PY@tok@sa\endcsname{\def\PY@tc##1{\textcolor[rgb]{0.73,0.13,0.13}{##1}}}
\expandafter\def\csname PY@tok@sb\endcsname{\def\PY@tc##1{\textcolor[rgb]{0.73,0.13,0.13}{##1}}}
\expandafter\def\csname PY@tok@sc\endcsname{\def\PY@tc##1{\textcolor[rgb]{0.73,0.13,0.13}{##1}}}
\expandafter\def\csname PY@tok@dl\endcsname{\def\PY@tc##1{\textcolor[rgb]{0.73,0.13,0.13}{##1}}}
\expandafter\def\csname PY@tok@s2\endcsname{\def\PY@tc##1{\textcolor[rgb]{0.73,0.13,0.13}{##1}}}
\expandafter\def\csname PY@tok@sh\endcsname{\def\PY@tc##1{\textcolor[rgb]{0.73,0.13,0.13}{##1}}}
\expandafter\def\csname PY@tok@s1\endcsname{\def\PY@tc##1{\textcolor[rgb]{0.73,0.13,0.13}{##1}}}
\expandafter\def\csname PY@tok@mb\endcsname{\def\PY@tc##1{\textcolor[rgb]{0.40,0.40,0.40}{##1}}}
\expandafter\def\csname PY@tok@mf\endcsname{\def\PY@tc##1{\textcolor[rgb]{0.40,0.40,0.40}{##1}}}
\expandafter\def\csname PY@tok@mh\endcsname{\def\PY@tc##1{\textcolor[rgb]{0.40,0.40,0.40}{##1}}}
\expandafter\def\csname PY@tok@mi\endcsname{\def\PY@tc##1{\textcolor[rgb]{0.40,0.40,0.40}{##1}}}
\expandafter\def\csname PY@tok@il\endcsname{\def\PY@tc##1{\textcolor[rgb]{0.40,0.40,0.40}{##1}}}
\expandafter\def\csname PY@tok@mo\endcsname{\def\PY@tc##1{\textcolor[rgb]{0.40,0.40,0.40}{##1}}}
\expandafter\def\csname PY@tok@ch\endcsname{\let\PY@it=\textit\def\PY@tc##1{\textcolor[rgb]{0.25,0.50,0.50}{##1}}}
\expandafter\def\csname PY@tok@cm\endcsname{\let\PY@it=\textit\def\PY@tc##1{\textcolor[rgb]{0.25,0.50,0.50}{##1}}}
\expandafter\def\csname PY@tok@cpf\endcsname{\let\PY@it=\textit\def\PY@tc##1{\textcolor[rgb]{0.25,0.50,0.50}{##1}}}
\expandafter\def\csname PY@tok@c1\endcsname{\let\PY@it=\textit\def\PY@tc##1{\textcolor[rgb]{0.25,0.50,0.50}{##1}}}
\expandafter\def\csname PY@tok@cs\endcsname{\let\PY@it=\textit\def\PY@tc##1{\textcolor[rgb]{0.25,0.50,0.50}{##1}}}

\def\PYZbs{\char`\\}
\def\PYZus{\char`\_}
\def\PYZob{\char`\{}
\def\PYZcb{\char`\}}
\def\PYZca{\char`\^}
\def\PYZam{\char`\&}
\def\PYZlt{\char`\<}
\def\PYZgt{\char`\>}
\def\PYZsh{\char`\#}
\def\PYZpc{\char`\%}
\def\PYZdl{\char`\$}
\def\PYZhy{\char`\-}
\def\PYZsq{\char`\'}
\def\PYZdq{\char`\"}
\def\PYZti{\char`\~}
% for compatibility with earlier versions
\def\PYZat{@}
\def\PYZlb{[}
\def\PYZrb{]}
\makeatother


    % Exact colors from NB
    \definecolor{incolor}{rgb}{0.0, 0.0, 0.5}
    \definecolor{outcolor}{rgb}{0.545, 0.0, 0.0}



    
    % Prevent overflowing lines due to hard-to-break entities
    \sloppy 
    % Setup hyperref package
    \hypersetup{
      breaklinks=true,  % so long urls are correctly broken across lines
      colorlinks=true,
      urlcolor=urlcolor,
      linkcolor=linkcolor,
      citecolor=citecolor,
      }
    % Slightly bigger margins than the latex defaults
    
    \geometry{verbose,tmargin=1in,bmargin=1in,lmargin=1in,rmargin=1in}
    
    

    \begin{document}
    
    
    \maketitle
    
    

    
    \hypertarget{introduction-to-computational-software}{%
\section{Introduction to Computational
Software}\label{introduction-to-computational-software}}

\href{cichos@physik.uni-leipzig.de}{Frank Cichos} (2019)

    

    Table of Contents{}

{{1~~}Introduction to Computational Software}

{{2~~}Lecture 2}

{{2.1~~}A Short Primer on Plotting}

{{2.2~~}Random numbers}

{{2.2.1~~}Uniformly distributed random numbers}

{{2.2.2~~}Normally distributed random numbers}

{{2.2.3~~}Random distribution of integers}

{{2.3~~}Input and Output}

{{2.3.1~~}Keyboard input}

{{2.3.2~~}Screen output}

{{2.4~~}File input/output}

{{2.4.1~~}Reading data from a text file}

{{2.4.2~~}Writing data to a text file}

{{2.4.3~~}Pandas Module}

{{2.5~~}Advanced Plotting}

{{2.5.1~~}Simple Plotting}

{{2.5.2~~}Error bars}

{{2.5.3~~}Setting plotting limits and excluding data}

{{2.5.4~~}Subplots}

{{2.6~~}Animations}

    \hypertarget{lecture-2}{%
\section{Lecture 2}\label{lecture-2}}

    \hypertarget{a-short-primer-on-plotting}{%
\subsection{A Short Primer on
Plotting}\label{a-short-primer-on-plotting}}

Before we start today, we give a short primer on plotting, since
plotting data is very useful tool to visualize data during our
exploration of python. We will go into more details of the plotting
functions in matplotlib later. Here we just concentrate on the basics.

The matplotlib environment is accessible in multiple ways and has a
large functionality. We will use the \emph{pyplot} functions in the
matplotlib. These can be loaded by the \emph{\%pylab} magic command. The
pylab functions can then be used without a namespace. If you prefer
using a namespace, you may import pylab function by

\begin{verbatim}
import matplotlib.pyplot as plt
\end{verbatim}

They are contained in matplotlib in the pyplot module functions. When
using the Jupyter environment, the plotting functions direct their
output into an extra window. If you want to keep your plots inside the
notebook, you my add the \textbf{\emph{inline}} keyword to the pylab
magic. Plotting is then directed to the notebook page and a preview is
even maintained when saving the notebook.

    \begin{Verbatim}[commandchars=\\\{\}]
{\color{incolor}In [{\color{incolor}2}]:} \PY{o}{\PYZpc{}}\PY{k}{pylab} inline
        \PY{c+c1}{\PYZsh{}\PYZpc{}matplotlib inline}
        \PY{c+c1}{\PYZsh{} you may also use }
        \PY{c+c1}{\PYZsh{} import matplotlib.pyplot als plt }
        \PY{c+c1}{\PYZsh{} plotting functions are then available with the namespace plt}
        
        \PY{k+kn}{import} \PY{n+nn}{numpy} \PY{k}{as} \PY{n+nn}{np}
        \PY{o}{\PYZpc{}}\PY{k}{config} InlineBackend.figure\PYZus{}format = \PYZsq{}retina\PYZsq{} 
        \PY{c+c1}{\PYZsh{} this is just for Macbooks to create higher resolution previews of the plots in the notebook}
        
        \PY{c+c1}{\PYZsh{} the lines below set a number of parameters for plotting, such as label font size, title font size, which you may find useful}
        \PY{n}{plt}\PY{o}{.}\PY{n}{rcParams}\PY{o}{.}\PY{n}{update}\PY{p}{(}\PY{p}{\PYZob{}}\PY{l+s+s1}{\PYZsq{}}\PY{l+s+s1}{font.size}\PY{l+s+s1}{\PYZsq{}}\PY{p}{:} \PY{l+m+mi}{12}\PY{p}{,}
                             \PY{l+s+s1}{\PYZsq{}}\PY{l+s+s1}{axes.titlesize}\PY{l+s+s1}{\PYZsq{}}\PY{p}{:} \PY{l+m+mi}{20}\PY{p}{,}
                             \PY{l+s+s1}{\PYZsq{}}\PY{l+s+s1}{axes.labelsize}\PY{l+s+s1}{\PYZsq{}}\PY{p}{:} \PY{l+m+mi}{18}\PY{p}{,}
                             \PY{l+s+s1}{\PYZsq{}}\PY{l+s+s1}{axes.labelpad}\PY{l+s+s1}{\PYZsq{}}\PY{p}{:} \PY{l+m+mi}{14}\PY{p}{,}
                             \PY{l+s+s1}{\PYZsq{}}\PY{l+s+s1}{lines.linewidth}\PY{l+s+s1}{\PYZsq{}}\PY{p}{:} \PY{l+m+mi}{1}\PY{p}{,}
                             \PY{l+s+s1}{\PYZsq{}}\PY{l+s+s1}{lines.markersize}\PY{l+s+s1}{\PYZsq{}}\PY{p}{:} \PY{l+m+mi}{10}\PY{p}{,}
                             \PY{l+s+s1}{\PYZsq{}}\PY{l+s+s1}{xtick.labelsize}\PY{l+s+s1}{\PYZsq{}} \PY{p}{:} \PY{l+m+mi}{16}\PY{p}{,}
                             \PY{l+s+s1}{\PYZsq{}}\PY{l+s+s1}{ytick.labelsize}\PY{l+s+s1}{\PYZsq{}} \PY{p}{:} \PY{l+m+mi}{16}\PY{p}{,}
                             \PY{l+s+s1}{\PYZsq{}}\PY{l+s+s1}{xtick.top}\PY{l+s+s1}{\PYZsq{}} \PY{p}{:} \PY{k+kc}{True}\PY{p}{,}
                             \PY{l+s+s1}{\PYZsq{}}\PY{l+s+s1}{xtick.direction}\PY{l+s+s1}{\PYZsq{}} \PY{p}{:} \PY{l+s+s1}{\PYZsq{}}\PY{l+s+s1}{in}\PY{l+s+s1}{\PYZsq{}}\PY{p}{,}
                             \PY{l+s+s1}{\PYZsq{}}\PY{l+s+s1}{ytick.right}\PY{l+s+s1}{\PYZsq{}} \PY{p}{:} \PY{k+kc}{True}\PY{p}{,}
                             \PY{l+s+s1}{\PYZsq{}}\PY{l+s+s1}{ytick.direction}\PY{l+s+s1}{\PYZsq{}} \PY{p}{:} \PY{l+s+s1}{\PYZsq{}}\PY{l+s+s1}{in}\PY{l+s+s1}{\PYZsq{}}\PY{p}{,}\PY{p}{\PYZcb{}}\PY{p}{)} 
\end{Verbatim}


    \begin{Verbatim}[commandchars=\\\{\}]
Populating the interactive namespace from numpy and matplotlib

    \end{Verbatim}

    The \emph{plot(x,y)} command can then be used to make a plot of data
stored in y list or an array. If no x-axis is provided, the index of the
array is used.

    \begin{Verbatim}[commandchars=\\\{\}]
{\color{incolor}In [{\color{incolor}7}]:} \PY{n}{t}\PY{o}{=}\PY{n}{np}\PY{o}{.}\PY{n}{linspace}\PY{p}{(}\PY{l+m+mi}{0}\PY{p}{,}\PY{l+m+mi}{2}\PY{o}{*}\PY{n}{np}\PY{o}{.}\PY{n}{pi}\PY{p}{,}\PY{l+m+mi}{20}\PY{p}{)} \PY{c+c1}{\PYZsh{} think about the linspace again}
        
        \PY{n}{plot}\PY{p}{(}\PY{n}{sin}\PY{p}{(}\PY{n}{t}\PY{p}{)}\PY{p}{)} \PY{c+c1}{\PYZsh{} use the index of the array as x\PYZhy{}axis}
        \PY{n}{show}\PY{p}{(}\PY{p}{)}
\end{Verbatim}


    \begin{center}
    \adjustimage{max size={0.9\linewidth}{0.9\paperheight}}{output_7_0.png}
    \end{center}
    { \hspace*{\fill} \\}
    
    \begin{Verbatim}[commandchars=\\\{\}]
{\color{incolor}In [{\color{incolor}8}]:} \PY{n}{plot}\PY{p}{(}\PY{n}{t}\PY{p}{,}\PY{n}{sin}\PY{p}{(}\PY{n}{t}\PY{p}{)}\PY{p}{)} \PY{c+c1}{\PYZsh{} using x\PYZhy{}axis}
        \PY{n}{xlabel}\PY{p}{(}\PY{l+s+s1}{\PYZsq{}}\PY{l+s+s1}{t}\PY{l+s+s1}{\PYZsq{}}\PY{p}{)} \PY{c+c1}{\PYZsh{} set the x\PYZhy{}axis label}
        \PY{n}{ylabel}\PY{p}{(}\PY{l+s+s1}{\PYZsq{}}\PY{l+s+s1}{y}\PY{l+s+s1}{\PYZsq{}}\PY{p}{)} \PY{c+c1}{\PYZsh{} set the y\PYZhy{}axis label}
        \PY{n}{show}\PY{p}{(}\PY{p}{)}
\end{Verbatim}


    \begin{center}
    \adjustimage{max size={0.9\linewidth}{0.9\paperheight}}{output_8_0.png}
    \end{center}
    { \hspace*{\fill} \\}
    
    If you prefer to use symbols for plotting just use the

\begin{verbatim}
scatter(x,y)
\end{verbatim}

command of pylab. Note that the scatter command requires a \emph{x} and
\emph{y} values and you can set the marker symbol (see an overview of
the \href{https://matplotlib.org/api/markers_api.html}{marker symbols}).

    \begin{Verbatim}[commandchars=\\\{\}]
{\color{incolor}In [{\color{incolor}9}]:} \PY{n}{scatter}\PY{p}{(}\PY{n}{t}\PY{p}{,}\PY{n}{sin}\PY{p}{(}\PY{n}{t}\PY{p}{)}\PY{p}{,}\PY{n}{marker}\PY{o}{=}\PY{l+s+s1}{\PYZsq{}}\PY{l+s+s1}{P}\PY{l+s+s1}{\PYZsq{}}\PY{p}{)}
        \PY{n}{xlabel}\PY{p}{(}\PY{l+s+s1}{\PYZsq{}}\PY{l+s+s1}{t}\PY{l+s+s1}{\PYZsq{}}\PY{p}{)} \PY{c+c1}{\PYZsh{} set the x\PYZhy{}axis label}
        \PY{n}{ylabel}\PY{p}{(}\PY{l+s+s1}{\PYZsq{}}\PY{l+s+s1}{y}\PY{l+s+s1}{\PYZsq{}}\PY{p}{)} \PY{c+c1}{\PYZsh{} set the y\PYZhy{}axis label}
        \PY{n}{show}\PY{p}{(}\PY{p}{)}
\end{Verbatim}


    \begin{center}
    \adjustimage{max size={0.9\linewidth}{0.9\paperheight}}{output_10_0.png}
    \end{center}
    { \hspace*{\fill} \\}
    
    A very useful plotting command is also the \emph{hist} command. It
generates a histogram of the data provided. If only the data is given,
bins are calculated automatically. If you supply an array of intervalls
with \emph{hist(data,bins=b)}, where \emph{b} is and array, the hist
command calculates the histogram for the supplied bins. \emph{normed=1}
normalizes the area below the histogram to 1. The hist command not only
returns the graph, but also the occurrences and bins used as shown in
the example.

    \begin{Verbatim}[commandchars=\\\{\}]
{\color{incolor}In [{\color{incolor}11}]:} \PY{n}{t}\PY{o}{=}\PY{n}{np}\PY{o}{.}\PY{n}{linspace}\PY{p}{(}\PY{l+m+mi}{0}\PY{p}{,}\PY{n}{np}\PY{o}{.}\PY{n}{pi}\PY{p}{,}\PY{l+m+mi}{1000}\PY{p}{)}
         \PY{n}{b}\PY{o}{=}\PY{n}{np}\PY{o}{.}\PY{n}{linspace}\PY{p}{(}\PY{o}{\PYZhy{}}\PY{l+m+mi}{2}\PY{p}{,}\PY{l+m+mi}{2}\PY{p}{,}\PY{l+m+mi}{50}\PY{p}{)}
         \PY{n}{n}\PY{p}{,} \PY{n}{bins}\PY{p}{,} \PY{n}{patches}\PY{o}{=}\PY{n}{hist}\PY{p}{(}\PY{n}{cos}\PY{p}{(}\PY{n}{t}\PY{p}{)}\PY{p}{,}\PY{n}{bins}\PY{o}{=}\PY{n}{b}\PY{p}{,}\PY{n}{normed}\PY{o}{=}\PY{l+m+mi}{1}\PY{p}{)}
         \PY{n}{xlabel}\PY{p}{(}\PY{l+s+s1}{\PYZsq{}}\PY{l+s+s1}{y}\PY{l+s+s1}{\PYZsq{}}\PY{p}{)} \PY{c+c1}{\PYZsh{} set the x\PYZhy{}axis label}
         \PY{n}{ylabel}\PY{p}{(}\PY{l+s+s1}{\PYZsq{}}\PY{l+s+s1}{occurrence}\PY{l+s+s1}{\PYZsq{}}\PY{p}{)} \PY{c+c1}{\PYZsh{} set the y\PYZhy{}axis label}
         \PY{n}{show}\PY{p}{(}\PY{p}{)}
\end{Verbatim}


    \begin{center}
    \adjustimage{max size={0.9\linewidth}{0.9\paperheight}}{output_12_0.png}
    \end{center}
    { \hspace*{\fill} \\}
    
    The example shown above actually has a physical meaning. Imagine you
have a mass on a spring which is at rest at a position \(x_{0}\) and at
\(t=0\) elongated by a distance \(\Delta x\). The spring will then start
to oscillate according to

\begin{equation}
x(t)=\Delta x \cos(\omega t)
\end{equation}

If you now need to calculate the probability to find the spring at a
certain elongation you need to realize that the oscillator spends a
different amount of time at different positions. The time \(dt\) spend
in the interval {[}\(x(t)\),\(x(t)+dx\){]} depends on the speed, i.e.

\begin{equation}
v(t)=\frac{dx}{dt}=-\omega \Delta x  \sin(\omega t)
\end{equation}

The probability to find the oscillator at a certain intervall then is
the fraction of time residing in this intervall normalized by the half
the oscillation period \(T/2\).

\begin{equation}
\frac{dt}{T/2}=\frac{1}{T/2}\frac{dx}{v(t)}=\frac{1}{T/2}\frac{-dx}{\omega \Delta x  \sin(\omega t)}
\end{equation}

As the frequency of the oscillator is \(\omega=2\pi/T\) we can replace
\(T\) by \(T=2\pi/\omega\) which yields

\begin{equation}
p(x)dx=\frac{1}{\pi \Delta x}\frac{dx}{\sqrt{1-\left (\frac{x(t)}{\Delta x}\right )^2}}
\end{equation}

This is the probability density of finding an oscillating spring at a
certain elongation \(x(t)\). If you look at the example more closely, it
tells you, that you find and elongation more likely when the speed of
the mass is slow. This is even a more general issue in non-equilibrium
physics. If cells or cars are moving with variable speed, they are more
likely to be found at places where they are slow.

    \begin{center}\rule{0.5\linewidth}{\linethickness}\end{center}

\hypertarget{random-numbers}{%
\subsection{Random numbers}\label{random-numbers}}

Jump to top

Random numbers are widely used in science and engineering computations.
They can be used to simulate noisy data, or to model physical phenomena
like the distribution of velocities of molecules in a gas, or to act
like the roll of dice in a game. Monte Carlo simulation techniques,
which will be part of a later theory lecture rely heavily and random
number. Processes like single photon emission or Brownian motion are
stochastic processes, with an intrisic randomness as well. But there are
even methods for numerically evaluating multi-dimensional integrals
using random numbers.

The basic idea of a random number generator is that it should be able to
produce a sequence of numbers that are distributed according to some
predetermined distribution function. NumPy provides a number of such
random number generators in its library numpy.random. Here we focus on
three:

-~rand/uniform -~randn - exponential - randint

    \hypertarget{uniformly-distributed-random-numbers---rand}{%
\subsubsection{Uniformly distributed random numbers -
rand()}\label{uniformly-distributed-random-numbers---rand}}

Jump to top

The rand(num) function creates an array of num floats uniformly
distributed on the interval from 0 to 1.

    \begin{Verbatim}[commandchars=\\\{\}]
{\color{incolor}In [{\color{incolor}12}]:} \PY{k+kn}{from} \PY{n+nn}{numpy}\PY{n+nn}{.}\PY{n+nn}{random} \PY{k}{import} \PY{o}{*}
\end{Verbatim}


    \begin{Verbatim}[commandchars=\\\{\}]
{\color{incolor}In [{\color{incolor}13}]:} \PY{n}{rand}\PY{p}{(}\PY{p}{)}
\end{Verbatim}


\begin{Verbatim}[commandchars=\\\{\}]
{\color{outcolor}Out[{\color{outcolor}13}]:} 0.33103926057838196
\end{Verbatim}
            
    If you supply the argument \emph{num} to the rand() function you obtain
an array of equally distributed numbers with \emph{num} elements.

    \begin{Verbatim}[commandchars=\\\{\}]
{\color{incolor}In [{\color{incolor}14}]:} \PY{n}{rand}\PY{p}{(}\PY{l+m+mi}{5}\PY{p}{)}
\end{Verbatim}


\begin{Verbatim}[commandchars=\\\{\}]
{\color{outcolor}Out[{\color{outcolor}14}]:} array([0.68315106, 0.18070324, 0.93086355, 0.12285188, 0.43561634])
\end{Verbatim}
            
    You may also obtain a multi-dimensional array if you give two or more
numbers to the the rand function.

    \begin{Verbatim}[commandchars=\\\{\}]
{\color{incolor}In [{\color{incolor}15}]:} \PY{n}{rand}\PY{p}{(}\PY{l+m+mi}{5}\PY{p}{,}\PY{l+m+mi}{2}\PY{p}{)}
\end{Verbatim}


\begin{Verbatim}[commandchars=\\\{\}]
{\color{outcolor}Out[{\color{outcolor}15}]:} array([[0.66020283, 0.37155731],
                [0.58697475, 0.12167188],
                [0.36114152, 0.11457844],
                [0.74493942, 0.18887414],
                [0.23439397, 0.20053688]])
\end{Verbatim}
            
    \begin{Verbatim}[commandchars=\\\{\}]
{\color{incolor}In [{\color{incolor}18}]:} \PY{n}{b}\PY{o}{=}\PY{n}{np}\PY{o}{.}\PY{n}{linspace}\PY{p}{(}\PY{l+m+mi}{0}\PY{p}{,}\PY{l+m+mi}{1}\PY{p}{,}\PY{l+m+mi}{50}\PY{p}{)}
         \PY{n}{hist}\PY{p}{(}\PY{n}{rand}\PY{p}{(}\PY{l+m+mi}{10000}\PY{p}{)}\PY{p}{,}\PY{n}{bins}\PY{o}{=}\PY{n}{b}\PY{p}{,}\PY{n}{normed}\PY{o}{=}\PY{l+m+mi}{1}\PY{p}{)}\PY{p}{;}
\end{Verbatim}


    \begin{center}
    \adjustimage{max size={0.9\linewidth}{0.9\paperheight}}{output_22_0.png}
    \end{center}
    { \hspace*{\fill} \\}
    
    \begin{center}\rule{0.5\linewidth}{\linethickness}\end{center}

\hypertarget{normally-distributed-random-numbers---randn}{%
\subsubsection{Normally distributed random numbers -
randn()}\label{normally-distributed-random-numbers---randn}}

    The function randn(num) produces a normal or Gaussian distribution of
num random numbers with a mean of 0 and a standard deviation of 1. They
are distributed according to

\begin{equation}
p(x)dx=\frac{1}{\sqrt{2\pi}}e^{-\frac{x^2}{2}}dx
\end{equation}

Similarly as all the other random number function, you may supply one or
multiple arguments to the \emph{rand()} function. The result is again a
multi-dimensional array of random numbers.

    \begin{Verbatim}[commandchars=\\\{\}]
{\color{incolor}In [{\color{incolor}12}]:} \PY{n}{randn}\PY{p}{(}\PY{l+m+mi}{10}\PY{p}{)}
\end{Verbatim}


\begin{Verbatim}[commandchars=\\\{\}]
{\color{outcolor}Out[{\color{outcolor}12}]:} array([-0.0802658 ,  0.0994643 , -0.14251669, -0.0700831 ,  0.35572245,
                -1.33922514, -1.16528455, -0.38371025,  0.65728619,  1.27304487])
\end{Verbatim}
            
    \textbf{Exercise}

Plot a histogram of the random numbers drawn from a normal distribution
as provided by the function \emph{randn}. Write a script calculating the
variance of the normally distributed random numbers using the numpy
function \emph{sum(data)}.

    \begin{Verbatim}[commandchars=\\\{\}]
{\color{incolor}In [{\color{incolor}368}]:} \PY{c+c1}{\PYZsh{} code goes here}
\end{Verbatim}


    \textbf{Exercise}

Write a short script which calculates normally distributed numbers with
a variance \(\sigma^2\) and a mean \(x_0\). Plot a histogram of \(10^6\)
random numbers using 50 bins. Calculate the mean and the variance of
\(10^6\) random numbers drawn from this distribution.

    \begin{Verbatim}[commandchars=\\\{\}]
{\color{incolor}In [{\color{incolor}12}]:} \PY{c+c1}{\PYZsh{}code goes here}
\end{Verbatim}


    The following lines create random numbers, which are distributed by a
normal distribution. You can use such normally distributed random
numbers to generate a random walk. Such a random walk is a simple
representation of Brownian motion of colloids suspended in a liquid. You
will use study a random walk in the current exercise and connect it to
the diffusion coefficient.

    \begin{Verbatim}[commandchars=\\\{\}]
{\color{incolor}In [{\color{incolor}40}]:} \PY{n}{x}\PY{p}{,}\PY{n}{y}\PY{o}{=}\PY{p}{[}\PY{n}{randn}\PY{p}{(}\PY{l+m+mi}{1000}\PY{p}{)}\PY{o}{.}\PY{n}{cumsum}\PY{p}{(}\PY{p}{)}\PY{p}{,}\PY{n}{randn}\PY{p}{(}\PY{l+m+mi}{1000}\PY{p}{)}\PY{o}{.}\PY{n}{cumsum}\PY{p}{(}\PY{p}{)}\PY{p}{]}
\end{Verbatim}


    \begin{Verbatim}[commandchars=\\\{\}]
{\color{incolor}In [{\color{incolor}41}]:} \PY{n}{figure}\PY{p}{(}\PY{n}{figsize}\PY{o}{=}\PY{p}{(}\PY{l+m+mi}{6}\PY{p}{,}\PY{l+m+mi}{6}\PY{p}{)}\PY{p}{)}
         \PY{n}{plot}\PY{p}{(}\PY{n}{x}\PY{p}{,}\PY{n}{y}\PY{p}{)}
         \PY{n}{xlabel}\PY{p}{(}\PY{l+s+s1}{\PYZsq{}}\PY{l+s+s1}{x}\PY{l+s+s1}{\PYZsq{}}\PY{p}{)} \PY{c+c1}{\PYZsh{} set the x\PYZhy{}axis label}
         \PY{n}{ylabel}\PY{p}{(}\PY{l+s+s1}{\PYZsq{}}\PY{l+s+s1}{y}\PY{l+s+s1}{\PYZsq{}}\PY{p}{)} \PY{c+c1}{\PYZsh{} set the y\PYZhy{}axis label}
         \PY{n}{show}\PY{p}{(}\PY{p}{)}
\end{Verbatim}


    \begin{center}
    \adjustimage{max size={0.9\linewidth}{0.9\paperheight}}{output_32_0.png}
    \end{center}
    { \hspace*{\fill} \\}
    
    Note that the above example uses the \emph{cumsum()} function. The
cumsum function of an array of number \([x_0,x_1,x_2,..,x_n]\) delivers
an array with a progressive sum of elements
\([x_0,x_0+x_1,x_0+x_1+x_2,...,x_0+...+x_n]\). The line

\begin{verbatim}
x,y=[randn(1000).cumsum(),randn(1000).cumsum()]
\end{verbatim}

is therefore all you need to generate a random walk in two dimensions.

    \begin{center}\rule{0.5\linewidth}{\linethickness}\end{center}

\hypertarget{exponentially-distributed-numbers}{%
\subsection{Exponentially distributed
numbers}\label{exponentially-distributed-numbers}}

A number of processes in physics reveal an exponential statistics. For
example the probability to find a molecule under gravity at a certain
height \(h\) is distributed by a Boltzmann law for a non-zero
temperature.

\begin{equation}
    p(h)dh=p_{0}e^{-\frac{m\cdot g\cdot h}{k_{\rm B}T}}dh
\end{equation}

On the other hand, the probability to emit a photon spontaneously after
a certain time \(t\) follwings the excited state preparation (two level
system) is also exponentially distributed.

\begin{equation}
p(t)dt=p_{0}e^{-\frac{t}{\tau}}
\end{equation}

Thus after each excitation i.e.~by a laser pulse, a molecule emits a
single photon with the probability \(p(t)\) and the whole exponential
character in the statistics is only appearing after repeating the
experiment several times.

The exponential distribution of \emph{numpy} can be supplied with two
numbers.

\begin{verbatim}
exponential(b, n)
\end{verbatim}

The parameter b is giving the decay parameter. The number n is optional
and giving the number of samples to be provided. The numbers are
distributed according to

\begin{equation}
\frac{1}{b}e^{-\frac{x}{b}}
\end{equation}

    \begin{Verbatim}[commandchars=\\\{\}]
{\color{incolor}In [{\color{incolor}13}]:} \PY{n}{exponential}\PY{p}{(}\PY{l+m+mi}{1}\PY{p}{)}
\end{Verbatim}


\begin{Verbatim}[commandchars=\\\{\}]
{\color{outcolor}Out[{\color{outcolor}13}]:} 0.6680016157067631
\end{Verbatim}
            
    \begin{Verbatim}[commandchars=\\\{\}]
{\color{incolor}In [{\color{incolor}14}]:} \PY{n}{exponential}\PY{p}{(}\PY{l+m+mi}{1}\PY{p}{,}\PY{l+m+mi}{10}\PY{p}{)}
\end{Verbatim}


\begin{Verbatim}[commandchars=\\\{\}]
{\color{outcolor}Out[{\color{outcolor}14}]:} array([ 0.36131721,  0.51306895,  0.0608441 ,  2.00774215,  2.47457833,
                 0.70622877,  1.38093438,  0.45698494,  0.53849219,  0.25727531])
\end{Verbatim}
            
    You may want to test the changes in the exponential distribution with
the parameter \textbf{b}.

    \begin{Verbatim}[commandchars=\\\{\}]
{\color{incolor}In [{\color{incolor}496}]:} \PY{n}{b}\PY{o}{=}\PY{n}{np}\PY{o}{.}\PY{n}{linspace}\PY{p}{(}\PY{l+m+mi}{0}\PY{p}{,}\PY{l+m+mi}{10}\PY{p}{,}\PY{l+m+mi}{50}\PY{p}{)}
          \PY{n}{hist}\PY{p}{(}\PY{n}{exponential}\PY{p}{(}\PY{l+m+mi}{1}\PY{p}{,}\PY{l+m+mi}{10000}\PY{p}{)}\PY{p}{,}\PY{n}{bins}\PY{o}{=}\PY{n}{b}\PY{p}{,}\PY{n}{normed}\PY{o}{=}\PY{l+m+mi}{1}\PY{p}{)}\PY{p}{;}
\end{Verbatim}


    \begin{center}
    \adjustimage{max size={0.9\linewidth}{0.9\paperheight}}{output_38_0.png}
    \end{center}
    { \hspace*{\fill} \\}
    
    \textbf{Exercise}

Calculate the mean value of the exponential distribution.

    \begin{Verbatim}[commandchars=\\\{\}]
{\color{incolor}In [{\color{incolor}499}]:} \PY{c+c1}{\PYZsh{} code goes here}
\end{Verbatim}


    \textbf{Exercise}

Construct a series of single photon events seperated by periods which
are exponentially distributed.

    \begin{Verbatim}[commandchars=\\\{\}]
{\color{incolor}In [{\color{incolor}501}]:} \PY{c+c1}{\PYZsh{} code goes here}
\end{Verbatim}


    \begin{center}\rule{0.5\linewidth}{\linethickness}\end{center}

\hypertarget{random-distribution-of-integers}{%
\subsubsection{Random distribution of
integers}\label{random-distribution-of-integers}}

The function randint(low, high, num) produces a uniform random
distribution of num integers between low (inclusive) and high
(exclusive).

    \begin{Verbatim}[commandchars=\\\{\}]
{\color{incolor}In [{\color{incolor}425}]:} \PY{n}{randint}\PY{p}{(}\PY{l+m+mi}{1}\PY{p}{,}\PY{l+m+mi}{10}\PY{p}{,}\PY{l+m+mi}{10}\PY{p}{)}
\end{Verbatim}


\begin{Verbatim}[commandchars=\\\{\}]
{\color{outcolor}Out[{\color{outcolor}425}]:} array([1, 6, 3, 4, 7, 4, 2, 4, 7, 2])
\end{Verbatim}
            
    \begin{Verbatim}[commandchars=\\\{\}]
{\color{incolor}In [{\color{incolor}426}]:} \PY{n}{items} \PY{o}{=} \PY{p}{[}\PY{l+m+mi}{1}\PY{p}{,} \PY{l+m+mi}{2}\PY{p}{,} \PY{l+m+mi}{3}\PY{p}{,} \PY{l+m+mi}{4}\PY{p}{,} \PY{l+m+mi}{5}\PY{p}{,} \PY{l+m+mi}{6}\PY{p}{,} \PY{l+m+mi}{7}\PY{p}{,} \PY{l+m+mi}{8}\PY{p}{,} \PY{l+m+mi}{9}\PY{p}{,} \PY{l+m+mi}{10}\PY{p}{]}
          \PY{n}{shuffle}\PY{p}{(}\PY{n}{items}\PY{p}{)}
          
          \PY{n+nb}{print}\PY{p}{(}\PY{n}{items}\PY{p}{)}
\end{Verbatim}


    \begin{Verbatim}[commandchars=\\\{\}]
[2, 1, 5, 10, 4, 6, 3, 9, 8, 7]

    \end{Verbatim}

    There are a number of other methods available in the random module of
numpy. Please refere to the
\href{https://docs.scipy.org/doc/numpy-1.13.0/reference/routines.random.html}{documentation}.

    \begin{center}\rule{0.5\linewidth}{\linethickness}\end{center}

\hypertarget{input-and-output}{%
\subsection{Input and Output}\label{input-and-output}}

    \hypertarget{keyboard-input}{%
\subsubsection{Keyboard input}\label{keyboard-input}}

Python has a function called \emph{input} for getting input from the
user and assigning it a variable name. It has the form

    \begin{Verbatim}[commandchars=\\\{\}]
{\color{incolor}In [{\color{incolor}429}]:} \PY{n}{value}\PY{o}{=}\PY{n+nb}{input}\PY{p}{(}\PY{l+s+s2}{\PYZdq{}}\PY{l+s+s2}{Tell me a number: }\PY{l+s+s2}{\PYZdq{}}\PY{p}{)}
          \PY{n+nb}{type}\PY{p}{(}\PY{n}{value}\PY{p}{)}
\end{Verbatim}


\begin{Verbatim}[commandchars=\\\{\}]
{\color{outcolor}Out[{\color{outcolor}429}]:} str
\end{Verbatim}
            
    The value contains the keyboard input as expected, but it is a string
(the u stands for ``uni-code'' which refers to the string coding system
Python uses). Because we want to use a number and not a string, we need
to convert it from a string to a number. We can do that with the eval
function by writing:

    \begin{Verbatim}[commandchars=\\\{\}]
{\color{incolor}In [{\color{incolor}430}]:} \PY{n}{v} \PY{o}{=} \PY{n+nb}{eval}\PY{p}{(}\PY{n}{value}\PY{p}{)}
          \PY{n+nb}{type}\PY{p}{(}\PY{n}{v}\PY{p}{)}
\end{Verbatim}


\begin{Verbatim}[commandchars=\\\{\}]
{\color{outcolor}Out[{\color{outcolor}430}]:} int
\end{Verbatim}
            
    \hypertarget{screen-output}{%
\subsubsection{Screen output}\label{screen-output}}

Screen output is possible by using the \emph{print} command. The
argument of the print function can be of different type. If you want to
achieve a specific format, you may want to use a formatted string. A
formatted string can be given to the \emph{print} function by
str.format() where str is a string that contains text that is written to
be the screen, as well as certain format specifiers contained in curly
braces \{\}. The format function contains the list of variables that are
to be printed. The commands in the input line below show the use of the
formating in different ways. Try to modify the script and observe the
resulting output.

    \begin{Verbatim}[commandchars=\\\{\}]
{\color{incolor}In [{\color{incolor}431}]:} \PY{n}{string1} \PY{o}{=} \PY{l+s+s2}{\PYZdq{}}\PY{l+s+s2}{How}\PY{l+s+s2}{\PYZdq{}}
          \PY{n}{string2} \PY{o}{=} \PY{l+s+s2}{\PYZdq{}}\PY{l+s+s2}{are you my friend?}\PY{l+s+s2}{\PYZdq{}}
          \PY{n}{int1} \PY{o}{=} \PY{l+m+mi}{34}
          \PY{n}{int2} \PY{o}{=} \PY{l+m+mi}{942885}
          \PY{n}{float1} \PY{o}{=} \PY{o}{\PYZhy{}}\PY{l+m+mf}{3.0}
          \PY{n}{float2} \PY{o}{=} \PY{l+m+mf}{3.141592653589793e\PYZhy{}14}
          \PY{n+nb}{print}\PY{p}{(}\PY{l+s+s1}{\PYZsq{}}\PY{l+s+s1}{ ***}\PY{l+s+s1}{\PYZsq{}}\PY{p}{)}
          
          \PY{n+nb}{print}\PY{p}{(}\PY{n}{string1}\PY{p}{)}
          \PY{n+nb}{print}\PY{p}{(}\PY{n}{string1} \PY{o}{+} \PY{l+s+s1}{\PYZsq{}}\PY{l+s+s1}{ }\PY{l+s+s1}{\PYZsq{}} \PY{o}{+} \PY{n}{string2}\PY{p}{)}
          \PY{n+nb}{print}\PY{p}{(}\PY{l+s+s1}{\PYZsq{}}\PY{l+s+s1}{ 1. }\PY{l+s+si}{\PYZob{}\PYZcb{}}\PY{l+s+s1}{ }\PY{l+s+si}{\PYZob{}\PYZcb{}}\PY{l+s+s1}{\PYZsq{}}\PY{o}{.}\PY{n}{format}\PY{p}{(}\PY{n}{string1}\PY{p}{,} \PY{n}{string2}\PY{p}{)}\PY{p}{)} 
          \PY{n+nb}{print}\PY{p}{(}\PY{l+s+s1}{\PYZsq{}}\PY{l+s+s1}{ 2. }\PY{l+s+si}{\PYZob{}0:s\PYZcb{}}\PY{l+s+s1}{ }\PY{l+s+si}{\PYZob{}1:s\PYZcb{}}\PY{l+s+s1}{\PYZsq{}}\PY{o}{.}\PY{n}{format}\PY{p}{(}\PY{n}{string1}\PY{p}{,} \PY{n}{string2}\PY{p}{)}\PY{p}{)}
          \PY{n+nb}{print}\PY{p}{(}\PY{l+s+s1}{\PYZsq{}}\PY{l+s+s1}{ 3. }\PY{l+s+si}{\PYZob{}0:s\PYZcb{}}\PY{l+s+s1}{ }\PY{l+s+si}{\PYZob{}0:s\PYZcb{}}\PY{l+s+s1}{ }\PY{l+s+si}{\PYZob{}1:s\PYZcb{}}\PY{l+s+s1}{ \PYZhy{} }\PY{l+s+si}{\PYZob{}0:s\PYZcb{}}\PY{l+s+s1}{ }\PY{l+s+si}{\PYZob{}1:s\PYZcb{}}\PY{l+s+s1}{\PYZsq{}}\PY{o}{.}\PY{n}{format}\PY{p}{(}\PY{n}{string1}\PY{p}{,} \PY{n}{string2}\PY{p}{)}\PY{p}{)} 
          \PY{n+nb}{print}\PY{p}{(}\PY{l+s+s1}{\PYZsq{}}\PY{l+s+s1}{ 4. }\PY{l+s+si}{\PYZob{}0:10s\PYZcb{}}\PY{l+s+si}{\PYZob{}1:5s\PYZcb{}}\PY{l+s+s1}{\PYZsq{}}\PY{o}{.}\PY{n}{format}\PY{p}{(}\PY{n}{string1}\PY{p}{,} \PY{n}{string2}\PY{p}{)}\PY{p}{)}
          \PY{n+nb}{print}\PY{p}{(}\PY{l+s+s1}{\PYZsq{}}\PY{l+s+s1}{ ***}\PY{l+s+s1}{\PYZsq{}}\PY{p}{)}
          \PY{n+nb}{print}\PY{p}{(}\PY{n}{int1}\PY{p}{,} \PY{n}{int2}\PY{p}{)}
          \PY{n+nb}{print}\PY{p}{(}\PY{l+s+s1}{\PYZsq{}}\PY{l+s+s1}{ 6. }\PY{l+s+si}{\PYZob{}0:d\PYZcb{}}\PY{l+s+s1}{ }\PY{l+s+si}{\PYZob{}1:d\PYZcb{}}\PY{l+s+s1}{\PYZsq{}}\PY{o}{.}\PY{n}{format}\PY{p}{(}\PY{n}{int1}\PY{p}{,} \PY{n}{int2}\PY{p}{)}\PY{p}{)} 
          \PY{n+nb}{print}\PY{p}{(}\PY{l+s+s1}{\PYZsq{}}\PY{l+s+s1}{ 7. }\PY{l+s+si}{\PYZob{}0:8d\PYZcb{}}\PY{l+s+s1}{ }\PY{l+s+si}{\PYZob{}1:10d\PYZcb{}}\PY{l+s+s1}{\PYZsq{}}\PY{o}{.}\PY{n}{format}\PY{p}{(}\PY{n}{int1}\PY{p}{,} \PY{n}{int2}\PY{p}{)}\PY{p}{)} 
          \PY{n+nb}{print}\PY{p}{(}\PY{l+s+s1}{\PYZsq{}}\PY{l+s+s1}{ ***}\PY{l+s+s1}{\PYZsq{}}\PY{p}{)}
          \PY{n+nb}{print}\PY{p}{(}\PY{l+s+s1}{\PYZsq{}}\PY{l+s+s1}{ 8. }\PY{l+s+si}{\PYZob{}0:0.3f\PYZcb{}}\PY{l+s+s1}{\PYZsq{}}\PY{o}{.}\PY{n}{format}\PY{p}{(}\PY{n}{float1}\PY{p}{)}\PY{p}{)}
          \PY{n+nb}{print}\PY{p}{(}\PY{l+s+s1}{\PYZsq{}}\PY{l+s+s1}{ 9. }\PY{l+s+si}{\PYZob{}0:6.3f\PYZcb{}}\PY{l+s+s1}{\PYZsq{}}\PY{o}{.}\PY{n}{format}\PY{p}{(}\PY{n}{float1}\PY{p}{)}\PY{p}{)} 
          \PY{n+nb}{print}\PY{p}{(}\PY{l+s+s1}{\PYZsq{}}\PY{l+s+s1}{10. }\PY{l+s+si}{\PYZob{}0:8.3f\PYZcb{}}\PY{l+s+s1}{\PYZsq{}}\PY{o}{.}\PY{n}{format}\PY{p}{(}\PY{n}{float1}\PY{p}{)}\PY{p}{)} 
          \PY{n+nb}{print}\PY{p}{(}\PY{l+m+mi}{2}\PY{o}{*}\PY{l+s+s1}{\PYZsq{}}\PY{l+s+s1}{ 11. }\PY{l+s+si}{\PYZob{}0:8.3f\PYZcb{}}\PY{l+s+s1}{\PYZsq{}}\PY{o}{.}\PY{n}{format}\PY{p}{(}\PY{n}{float1}\PY{p}{)}\PY{p}{)}
          \PY{n+nb}{print}\PY{p}{(}\PY{l+s+s1}{\PYZsq{}}\PY{l+s+s1}{ ***}\PY{l+s+s1}{\PYZsq{}}\PY{p}{)}
          \PY{n+nb}{print}\PY{p}{(}\PY{l+s+s1}{\PYZsq{}}\PY{l+s+s1}{12. }\PY{l+s+si}{\PYZob{}0:0.3e\PYZcb{}}\PY{l+s+s1}{\PYZsq{}}\PY{o}{.}\PY{n}{format}\PY{p}{(}\PY{n}{float2}\PY{p}{)}\PY{p}{)} 
          \PY{n+nb}{print}\PY{p}{(}\PY{l+s+s1}{\PYZsq{}}\PY{l+s+s1}{13. }\PY{l+s+si}{\PYZob{}0:10.3e\PYZcb{}}\PY{l+s+s1}{\PYZsq{}}\PY{o}{.}\PY{n}{format}\PY{p}{(}\PY{n}{float2}\PY{p}{)}\PY{p}{)} 
          \PY{n+nb}{print}\PY{p}{(}\PY{l+s+s1}{\PYZsq{}}\PY{l+s+s1}{14. }\PY{l+s+si}{\PYZob{}0:10.3f\PYZcb{}}\PY{l+s+s1}{\PYZsq{}}\PY{o}{.}\PY{n}{format}\PY{p}{(}\PY{n}{float2}\PY{p}{)}\PY{p}{)}
          \PY{n+nb}{print}\PY{p}{(}\PY{l+s+s1}{\PYZsq{}}\PY{l+s+s1}{ ***}\PY{l+s+s1}{\PYZsq{}}\PY{p}{)}
          \PY{n+nb}{print}\PY{p}{(}\PY{l+s+s1}{\PYZsq{}}\PY{l+s+s1}{15. 12345678901234567890}\PY{l+s+s1}{\PYZsq{}}\PY{p}{)}
          \PY{n+nb}{print}\PY{p}{(}\PY{l+s+s1}{\PYZsq{}}\PY{l+s+s1}{16. }\PY{l+s+si}{\PYZob{}0:s\PYZcb{}}\PY{l+s+s1}{\PYZhy{}\PYZhy{}}\PY{l+s+si}{\PYZob{}1:8d\PYZcb{}}\PY{l+s+s1}{,}\PY{l+s+si}{\PYZob{}2:10.3e\PYZcb{}}\PY{l+s+s1}{\PYZsq{}}\PY{o}{.}\PY{n}{format}\PY{p}{(}\PY{n}{string2}\PY{p}{,} \PY{n}{int1}\PY{p}{,} \PY{n}{float2}\PY{p}{)}\PY{p}{)}
\end{Verbatim}


    \begin{Verbatim}[commandchars=\\\{\}]
 ***
How
How are you my friend?
 1. How are you my friend?
 2. How are you my friend?
 3. How How are you my friend? - How are you my friend?
 4. How       are you my friend?
 ***
34 942885
 6. 34 942885
 7.       34     942885
 ***
 8. -3.000
 9. -3.000
10.   -3.000
 11.   -3.000 11.   -3.000
 ***
12. 3.142e-14
13.  3.142e-14
14.      0.000
 ***
15. 12345678901234567890
16. are you my friend?--      34, 3.142e-14

    \end{Verbatim}

    \begin{center}\rule{0.5\linewidth}{\linethickness}\end{center}

\hypertarget{file-inputoutput}{%
\subsection{File input/output}\label{file-inputoutput}}

    \hypertarget{reading-data-from-a-text-file}{%
\subsubsection{Reading data from a text
file}\label{reading-data-from-a-text-file}}

Often you would like to analyze data that you have stored in a text
file. Consider, for example, the data file below for an experiment
measuring the free fall of a mass.

\begin{verbatim}
Data for falling mass experiment
Date: 16-Aug-2013
Data taken by Frank and Ralf
data point  time (sec)  height (mm) uncertainty (mm)
0       0.0     180     3.5
1       0.5     182     4.5
2       1.0     178     4.0
3       1.5     165     5.5
4       2.0     160     2.5
5       2.5     148     3.0
6       3.0     136     2.5
\end{verbatim}

We would like to read these data into a Python program, associating the
data in each column with an appropriately named array. While there are a
multitude of ways to do this in Python, the simplest by far is to use
the NumPy
\href{https://docs.scipy.org/doc/numpy-1.14.0/reference/generated/numpy.loadtxt.html}{loadtxt}
function, whose use we illustrate here. Suppose that the name of the
text file is \textbf{MyData.txt}. Then we can read the data into four
different arrays with the following statement

    \begin{Verbatim}[commandchars=\\\{\}]
{\color{incolor}In [{\color{incolor}433}]:} \PY{n}{dataPt}\PY{p}{,} \PY{n}{time}\PY{p}{,} \PY{n}{height}\PY{p}{,} \PY{n}{error} \PY{o}{=} \PY{n}{np}\PY{o}{.}\PY{n}{loadtxt}\PY{p}{(}\PY{l+s+s2}{\PYZdq{}}\PY{l+s+s2}{MyData.txt}\PY{l+s+s2}{\PYZdq{}}\PY{p}{,} \PY{n}{skiprows}\PY{o}{=}\PY{l+m+mi}{5} \PY{p}{,} \PY{n}{unpack}\PY{o}{=}\PY{k+kc}{True}\PY{p}{)}
\end{Verbatim}


    If you don't want to read in all the columns of data, you can specify
which columns to read in using the usecols key word. For example, the
call

    \begin{Verbatim}[commandchars=\\\{\}]
{\color{incolor}In [{\color{incolor}434}]:} \PY{n}{time}\PY{p}{,} \PY{n}{height} \PY{o}{=} \PY{n}{np}\PY{o}{.}\PY{n}{loadtxt}\PY{p}{(}\PY{l+s+s2}{\PYZdq{}}\PY{l+s+s2}{MyData.txt}\PY{l+s+s2}{\PYZdq{}}\PY{p}{,} \PY{n}{skiprows}\PY{o}{=}\PY{l+m+mi}{5} \PY{p}{,} \PY{n}{usecols} \PY{o}{=} \PY{p}{(}\PY{l+m+mi}{1}\PY{p}{,}\PY{l+m+mi}{2}\PY{p}{)}\PY{p}{,} \PY{n}{unpack}\PY{o}{=}\PY{k+kc}{True}\PY{p}{)}
\end{Verbatim}


    reads in only columns 1 and 2; columns 0 and 3 are skipped.

    \hypertarget{writing-data-to-a-text-file}{%
\subsubsection{Writing data to a text
file}\label{writing-data-to-a-text-file}}

    There are plenty of ways to write data to a data file in Python. We will
stick to one very simple one that's suitable for writing data files in
text format. It uses the NumPy
\href{https://docs.scipy.org/doc/numpy-1.14.0/reference/generated/numpy.savetxt.html}{savetxt}
routine, which is the counterpart of the loadtxt routine introduced in
the previous section. The general form of the routine is

\begin{verbatim}
savetxt(filename, array, fmt="%0.18e", delimiter=" ", newline="\n", header="", footer="", comments="# ")
\end{verbatim}

We illustrate savetext below with a script that first creates four
arrays by reading in the data file MyData.txt, as discussed in the
previous section, and then writes that same data set to another file
MyDataOut.txt.

    \begin{Verbatim}[commandchars=\\\{\}]
{\color{incolor}In [{\color{incolor}435}]:} \PY{n}{dataPt}\PY{p}{,} \PY{n}{time}\PY{p}{,} \PY{n}{height}\PY{p}{,} \PY{n}{error} \PY{o}{=} \PY{n}{np}\PY{o}{.}\PY{n}{loadtxt}\PY{p}{(}\PY{l+s+s2}{\PYZdq{}}\PY{l+s+s2}{MyData.txt}\PY{l+s+s2}{\PYZdq{}}\PY{p}{,} \PY{n}{skiprows}\PY{o}{=}\PY{l+m+mi}{5} \PY{p}{,} \PY{n}{unpack}\PY{o}{=}\PY{k+kc}{True}\PY{p}{)}
\end{Verbatim}


    \begin{Verbatim}[commandchars=\\\{\}]
{\color{incolor}In [{\color{incolor}436}]:} \PY{n}{np}\PY{o}{.}\PY{n}{savetxt}\PY{p}{(}\PY{l+s+s1}{\PYZsq{}}\PY{l+s+s1}{MyDataOut.txt}\PY{l+s+s1}{\PYZsq{}}\PY{p}{,}\PY{n+nb}{list}\PY{p}{(}\PY{n+nb}{zip}\PY{p}{(}\PY{n}{dataPt}\PY{p}{,} \PY{n}{time}\PY{p}{,} \PY{n}{height}\PY{p}{,} \PY{n}{error}\PY{p}{)}\PY{p}{)}\PY{p}{,} \PY{n}{fmt}\PY{o}{=}\PY{l+s+s2}{\PYZdq{}}\PY{l+s+si}{\PYZpc{}12.1f}\PY{l+s+s2}{\PYZdq{}}\PY{p}{)}
\end{Verbatim}


    \textbf{Exercise}

Create 10 different random walks in 2-dimensions. Assemble the x,y
coordinates with the zip command and write it to a text file so that we
can use it later for plotting.

    \begin{Verbatim}[commandchars=\\\{\}]
{\color{incolor}In [{\color{incolor}437}]:} \PY{c+c1}{\PYZsh{} code goes here}
\end{Verbatim}


    \hypertarget{pandas-module}{%
\subsubsection{Pandas Module}\label{pandas-module}}

    Pandas is a software library written for the Python programming
language. It is used for data manipulation and analysis. It provides
special data structures and operations for the manipulation of numerical
tables and time series and builds on top of numpy.

\emph{~~Easy handling of missing data }~ Intelligent label-based
slicing, fancy indexing, and subsetting of large data sets

The data formats provided by the pandas module are used by several other
modules, such as the
\href{https://soft-matter.github.io/trackpy/v0.3.2/}{trackpy} which is a
moduly for feature tracking and analysis in image series.

    \begin{Verbatim}[commandchars=\\\{\}]
{\color{incolor}In [{\color{incolor}438}]:} \PY{k+kn}{import} \PY{n+nn}{pandas} \PY{k}{as} \PY{n+nn}{pd} \PY{c+c1}{\PYZsh{} import the pandas module}
          \PY{k+kn}{import} \PY{n+nn}{numpy} \PY{k}{as} \PY{n+nn}{np}
          \PY{o}{\PYZpc{}}\PY{k}{matplotlib} inline   
\end{Verbatim}


    Pandas provides two data structures

\begin{itemize}
\tightlist
\item
  Series
\item
  Data Frames
\end{itemize}

    A
\textbf{\href{https://pandas.pydata.org/pandas-docs/stable/generated/pandas.Series.html}{Series}}
is a one-dimensional labeled array capable of holding any data type
(integers, strings, floating point numbers, Python objects, etc.). The
axis labels are collectively referred to as the index.

    \begin{Verbatim}[commandchars=\\\{\}]
{\color{incolor}In [{\color{incolor}439}]:} \PY{n}{my\PYZus{}simple\PYZus{}series} \PY{o}{=} \PY{n}{pd}\PY{o}{.}\PY{n}{Series}\PY{p}{(}\PY{n}{np}\PY{o}{.}\PY{n}{random}\PY{o}{.}\PY{n}{randn}\PY{p}{(}\PY{l+m+mi}{7}\PY{p}{)}\PY{p}{,} \PY{n}{index}\PY{o}{=}\PY{p}{[}\PY{l+s+s1}{\PYZsq{}}\PY{l+s+s1}{a}\PY{l+s+s1}{\PYZsq{}}\PY{p}{,} \PY{l+s+s1}{\PYZsq{}}\PY{l+s+s1}{b}\PY{l+s+s1}{\PYZsq{}}\PY{p}{,} \PY{l+s+s1}{\PYZsq{}}\PY{l+s+s1}{c}\PY{l+s+s1}{\PYZsq{}}\PY{p}{,} \PY{l+s+s1}{\PYZsq{}}\PY{l+s+s1}{d}\PY{l+s+s1}{\PYZsq{}}\PY{p}{,} \PY{l+s+s1}{\PYZsq{}}\PY{l+s+s1}{e}\PY{l+s+s1}{\PYZsq{}}\PY{p}{,}\PY{l+s+s1}{\PYZsq{}}\PY{l+s+s1}{f}\PY{l+s+s1}{\PYZsq{}}\PY{p}{,}\PY{l+s+s1}{\PYZsq{}}\PY{l+s+s1}{g}\PY{l+s+s1}{\PYZsq{}}\PY{p}{]}\PY{p}{)}
          \PY{n}{my\PYZus{}simple\PYZus{}series}\PY{o}{.}\PY{n}{head}\PY{p}{(}\PY{p}{)}
\end{Verbatim}


\begin{Verbatim}[commandchars=\\\{\}]
{\color{outcolor}Out[{\color{outcolor}439}]:} a   -1.091345
          b   -0.482831
          c    0.738757
          d   -0.964868
          e   -1.454059
          dtype: float64
\end{Verbatim}
            
    There is a whole lot of functionality built into pandas data types. You
may of course also obtain the same functionality using numpy commands,
but you may find the pandas abbrevations very useful.

    \begin{Verbatim}[commandchars=\\\{\}]
{\color{incolor}In [{\color{incolor}440}]:} \PY{n}{my\PYZus{}simple\PYZus{}series}\PY{o}{.}\PY{n}{agg}\PY{p}{(}\PY{p}{[}\PY{l+s+s1}{\PYZsq{}}\PY{l+s+s1}{min}\PY{l+s+s1}{\PYZsq{}}\PY{p}{,}\PY{l+s+s1}{\PYZsq{}}\PY{l+s+s1}{max}\PY{l+s+s1}{\PYZsq{}}\PY{p}{,}\PY{l+s+s1}{\PYZsq{}}\PY{l+s+s1}{sum}\PY{l+s+s1}{\PYZsq{}}\PY{p}{,}\PY{l+s+s1}{\PYZsq{}}\PY{l+s+s1}{mean}\PY{l+s+s1}{\PYZsq{}}\PY{p}{]}\PY{p}{)} \PY{c+c1}{\PYZsh{} aggregate a number of properties into a single array}
\end{Verbatim}


\begin{Verbatim}[commandchars=\\\{\}]
{\color{outcolor}Out[{\color{outcolor}440}]:} min    -1.454059
          max     0.738757
          sum    -3.645023
          mean   -0.520718
          dtype: float64
\end{Verbatim}
            
    A
\textbf{\href{https://pandas.pydata.org/pandas-docs/stable/api.html\#dataframe}{DataFrame}}
is a two-dimensional size-mutable, potentially heterogeneous tabular
data structure with labeled axes (rows and columns). The example below
shows how such a DataFrame can be generated from the scratch. In
addition to the data supplied to the DataFrame method, an index column
is generated when creating a DataFrame. As in the case of \emph{Series}
there is a whole lot of functionality integrated into the DataFrame data
type which you may explore on the website.

    \begin{Verbatim}[commandchars=\\\{\}]
{\color{incolor}In [{\color{incolor}441}]:} \PY{n}{df} \PY{o}{=} \PY{n}{pd}\PY{o}{.}\PY{n}{DataFrame}\PY{p}{(}\PY{n}{np}\PY{o}{.}\PY{n}{random}\PY{o}{.}\PY{n}{randint}\PY{p}{(}\PY{n}{low}\PY{o}{=}\PY{l+m+mi}{0}\PY{p}{,} \PY{n}{high}\PY{o}{=}\PY{l+m+mi}{10}\PY{p}{,} \PY{n}{size}\PY{o}{=}\PY{p}{(}\PY{l+m+mi}{5}\PY{p}{,} \PY{l+m+mi}{5}\PY{p}{)}\PY{p}{)}\PY{p}{,}\PY{n}{columns}\PY{o}{=}\PY{p}{[}\PY{l+s+s1}{\PYZsq{}}\PY{l+s+s1}{column 1}\PY{l+s+s1}{\PYZsq{}}\PY{p}{,} \PY{l+s+s1}{\PYZsq{}}\PY{l+s+s1}{column 2}\PY{l+s+s1}{\PYZsq{}}\PY{p}{,} \PY{l+s+s1}{\PYZsq{}}\PY{l+s+s1}{columns 3}\PY{l+s+s1}{\PYZsq{}}\PY{p}{,} \PY{l+s+s1}{\PYZsq{}}\PY{l+s+s1}{column 4}\PY{l+s+s1}{\PYZsq{}}\PY{p}{,} \PY{l+s+s1}{\PYZsq{}}\PY{l+s+s1}{column 5}\PY{l+s+s1}{\PYZsq{}}\PY{p}{]}\PY{p}{)}
          \PY{n}{df}\PY{o}{.}\PY{n}{head}\PY{p}{(}\PY{p}{)}
\end{Verbatim}


\begin{Verbatim}[commandchars=\\\{\}]
{\color{outcolor}Out[{\color{outcolor}441}]:}    column 1  column 2  columns 3  column 4  column 5
          0         7         7          2         3         3
          1         9         4          0         4         9
          2         4         5          7         9         6
          3         0         4          4         2         2
          4         8         1          1         9         7
\end{Verbatim}
            
    Due to the labelling of the columns, each column may be accessed by its
column label. Labeling by names improves readability considerably.

    \begin{Verbatim}[commandchars=\\\{\}]
{\color{incolor}In [{\color{incolor}442}]:} \PY{n}{df}\PY{p}{[}\PY{l+s+s1}{\PYZsq{}}\PY{l+s+s1}{column 1}\PY{l+s+s1}{\PYZsq{}}\PY{p}{]}
\end{Verbatim}


\begin{Verbatim}[commandchars=\\\{\}]
{\color{outcolor}Out[{\color{outcolor}442}]:} 0    7
          1    9
          2    4
          3    0
          4    8
          Name: column 1, dtype: int64
\end{Verbatim}
            
    If you don't like this format, you can always return to a simple numpy
array with the \emph{as\_matrix()} method.

    \begin{Verbatim}[commandchars=\\\{\}]
{\color{incolor}In [{\color{incolor}443}]:} \PY{n}{df}\PY{o}{.}\PY{n}{as\PYZus{}matrix}\PY{p}{(}\PY{p}{)}
\end{Verbatim}


\begin{Verbatim}[commandchars=\\\{\}]
{\color{outcolor}Out[{\color{outcolor}443}]:} array([[7, 7, 2, 3, 3],
                 [9, 4, 0, 4, 9],
                 [4, 5, 7, 9, 6],
                 [0, 4, 4, 2, 2],
                 [8, 1, 1, 9, 7]])
\end{Verbatim}
            
    DataFrames may also be populated by text files such as comma separated
value files (short \emph{.csv}). These files contain data in text format
but also a column label, which can be read by the pandas method
\emph{read\_csv()}. You can find an example below, which reads the data
from the dust sensor on my balcony from April, 11th. You see the
different columns, where \textbf{P1} and \textbf{P2} correspond to the
\textbf{PM10} and \textbf{PM2.5} dust values in \(\mu g/m^3\).

    \begin{Verbatim}[commandchars=\\\{\}]
{\color{incolor}In [{\color{incolor}444}]:} \PY{n}{data} \PY{o}{=} \PY{n}{pd}\PY{o}{.}\PY{n}{DataFrame}\PY{p}{(}\PY{p}{)}
          \PY{n}{data} \PY{o}{=} \PY{n}{pd}\PY{o}{.}\PY{n}{read\PYZus{}csv}\PY{p}{(}\PY{l+s+s2}{\PYZdq{}}\PY{l+s+s2}{2018\PYZhy{}04\PYZhy{}11\PYZus{}sds011\PYZus{}sensor\PYZus{}12253.csv}\PY{l+s+s2}{\PYZdq{}}\PY{p}{,}\PY{n}{delimiter}\PY{o}{=}\PY{l+s+s2}{\PYZdq{}}\PY{l+s+s2}{;}\PY{l+s+s2}{\PYZdq{}}\PY{p}{,}\PY{n}{parse\PYZus{}dates}\PY{o}{=}\PY{k+kc}{False}\PY{p}{)}
          \PY{n}{data}\PY{o}{.}\PY{n}{head}\PY{p}{(}\PY{p}{)}
\end{Verbatim}


\begin{Verbatim}[commandchars=\\\{\}]
{\color{outcolor}Out[{\color{outcolor}444}]:}    sensor\_id sensor\_type  location     lat    lon            timestamp     P1  \textbackslash{}
          0      12253      SDS011      6189  52.527  13.39  2018-04-11T00:01:58  25.87   
          1      12253      SDS011      6189  52.527  13.39  2018-04-11T00:04:24  25.63   
          2      12253      SDS011      6189  52.527  13.39  2018-04-11T00:06:55  26.30   
          3      12253      SDS011      6189  52.527  13.39  2018-04-11T00:09:23  24.60   
          4      12253      SDS011      6189  52.527  13.39  2018-04-11T00:11:51  25.17   
          
             durP1  ratioP1     P2  durP2  ratioP2  
          0    NaN      NaN  19.37    NaN      NaN  
          1    NaN      NaN  20.53    NaN      NaN  
          2    NaN      NaN  22.00    NaN      NaN  
          3    NaN      NaN  20.30    NaN      NaN  
          4    NaN      NaN  20.23    NaN      NaN  
\end{Verbatim}
            
    \begin{Verbatim}[commandchars=\\\{\}]
{\color{incolor}In [{\color{incolor}445}]:} \PY{n}{data}\PY{p}{[}\PY{l+s+s1}{\PYZsq{}}\PY{l+s+s1}{P2}\PY{l+s+s1}{\PYZsq{}}\PY{p}{]}\PY{o}{.}\PY{n}{plot}\PY{p}{(}\PY{p}{)}
\end{Verbatim}


\begin{Verbatim}[commandchars=\\\{\}]
{\color{outcolor}Out[{\color{outcolor}445}]:} <matplotlib.axes.\_subplots.AxesSubplot at 0x127697550>
\end{Verbatim}
            
    \begin{center}
    \adjustimage{max size={0.9\linewidth}{0.9\paperheight}}{output_82_1.png}
    \end{center}
    { \hspace*{\fill} \\}
    
    \begin{Verbatim}[commandchars=\\\{\}]
{\color{incolor}In [{\color{incolor}446}]:} \PY{p}{(}\PY{n}{data}\PY{p}{[}\PY{l+s+s1}{\PYZsq{}}\PY{l+s+s1}{P1}\PY{l+s+s1}{\PYZsq{}}\PY{p}{]}\PY{o}{/}\PY{n}{data}\PY{p}{[}\PY{l+s+s1}{\PYZsq{}}\PY{l+s+s1}{P2}\PY{l+s+s1}{\PYZsq{}}\PY{p}{]}\PY{p}{)}\PY{o}{.}\PY{n}{plot}\PY{p}{(}\PY{p}{)}
\end{Verbatim}


\begin{Verbatim}[commandchars=\\\{\}]
{\color{outcolor}Out[{\color{outcolor}446}]:} <matplotlib.axes.\_subplots.AxesSubplot at 0x113d5a390>
\end{Verbatim}
            
    \begin{center}
    \adjustimage{max size={0.9\linewidth}{0.9\paperheight}}{output_83_1.png}
    \end{center}
    { \hspace*{\fill} \\}
    
    \begin{center}\rule{0.5\linewidth}{\linethickness}\end{center}

\hypertarget{advanced-plotting}{%
\subsection{Advanced Plotting}\label{advanced-plotting}}

The graphical representation of data---plotting---is one of the most
important tools for evaluating and understanding scientific data and
theoretical predictions. However, plot- ting is not a part of core
Python but is provided through one of several possible library modules.
The most highly developed and widely used plotting package for Python is
MatPlotLib (http://MatPlotLib.sourceforge.net/). It is a powerful and
flexible program that has become the de facto standard for 2-d plotting
with Python.

Because MatPlotLib is an external library---in fact it's a collection of
libraries---it must be imported into any routine that uses it.
MatPlotLib makes extensive use of NumPy so the two should be imported
together. Therefore, for any program for which you would like to produce
2-d plots, you should include the lines

\begin{verbatim}
import numpy as np
import matplotlib.pyplot as plt
\end{verbatim}

At the beginning of this lecture we used the \emph{pylab} magic for
plotting.

    \hypertarget{simple-plotting}{%
\subsubsection{Simple Plotting}\label{simple-plotting}}

    \begin{Verbatim}[commandchars=\\\{\}]
{\color{incolor}In [{\color{incolor}447}]:} \PY{k+kn}{import} \PY{n+nn}{numpy} \PY{k}{as} \PY{n+nn}{np}
          \PY{k+kn}{import} \PY{n+nn}{matplotlib}\PY{n+nn}{.}\PY{n+nn}{pyplot} \PY{k}{as} \PY{n+nn}{plt} 
          \PY{n}{x} \PY{o}{=} \PY{n}{np}\PY{o}{.}\PY{n}{linspace}\PY{p}{(}\PY{l+m+mi}{0}\PY{p}{,} \PY{l+m+mf}{4.}\PY{o}{*}\PY{n}{np}\PY{o}{.}\PY{n}{pi}\PY{p}{,} \PY{l+m+mi}{33}\PY{p}{)}
          \PY{n}{y} \PY{o}{=} \PY{n}{np}\PY{o}{.}\PY{n}{sin}\PY{p}{(}\PY{n}{x}\PY{p}{)}
          
          \PY{n}{plt}\PY{o}{.}\PY{n}{plot}\PY{p}{(}\PY{n}{x}\PY{p}{,} \PY{n}{y}\PY{p}{)}
          \PY{n}{plt}\PY{o}{.}\PY{n}{show}\PY{p}{(}\PY{p}{)}
\end{Verbatim}


    \begin{center}
    \adjustimage{max size={0.9\linewidth}{0.9\paperheight}}{output_86_0.png}
    \end{center}
    { \hspace*{\fill} \\}
    
    \begin{Verbatim}[commandchars=\\\{\}]
{\color{incolor}In [{\color{incolor}448}]:} \PY{k+kn}{import} \PY{n+nn}{numpy} \PY{k}{as} \PY{n+nn}{np}
          \PY{k+kn}{import} \PY{n+nn}{matplotlib}\PY{n+nn}{.}\PY{n+nn}{pyplot} \PY{k}{as} \PY{n+nn}{plt}
          \PY{c+c1}{\PYZsh{} read data from file}
          \PY{n}{xdata}\PY{p}{,} \PY{n}{ydata} \PY{o}{=} \PY{n}{np}\PY{o}{.}\PY{n}{loadtxt}\PY{p}{(}\PY{l+s+s1}{\PYZsq{}}\PY{l+s+s1}{wavePulseData.txt}\PY{l+s+s1}{\PYZsq{}}\PY{p}{,} \PY{n}{unpack}\PY{o}{=}\PY{k+kc}{True}\PY{p}{)}
          \PY{c+c1}{\PYZsh{} create x and y arrays for theory}
          \PY{n}{x} \PY{o}{=} \PY{n}{np}\PY{o}{.}\PY{n}{linspace}\PY{p}{(}\PY{o}{\PYZhy{}}\PY{l+m+mf}{10.}\PY{p}{,} \PY{l+m+mf}{10.}\PY{p}{,} \PY{l+m+mi}{200}\PY{p}{)}
          \PY{n}{y} \PY{o}{=} \PY{n}{np}\PY{o}{.}\PY{n}{sin}\PY{p}{(}\PY{n}{x}\PY{p}{)} \PY{o}{*} \PY{n}{np}\PY{o}{.}\PY{n}{exp}\PY{p}{(}\PY{o}{\PYZhy{}}\PY{p}{(}\PY{n}{x}\PY{o}{/}\PY{l+m+mf}{4.47}\PY{p}{)}\PY{o}{*}\PY{o}{*}\PY{l+m+mi}{2}\PY{p}{)}
          
          \PY{c+c1}{\PYZsh{} create plot}
          \PY{n}{plt}\PY{o}{.}\PY{n}{figure}\PY{p}{(}\PY{l+m+mi}{1}\PY{p}{,} \PY{n}{figsize} \PY{o}{=} \PY{p}{(}\PY{l+m+mi}{8}\PY{p}{,}\PY{l+m+mi}{6}\PY{p}{)} \PY{p}{)}
          \PY{n}{plt}\PY{o}{.}\PY{n}{plot}\PY{p}{(}\PY{n}{x}\PY{p}{,} \PY{n}{y}\PY{p}{,} \PY{l+s+s1}{\PYZsq{}}\PY{l+s+s1}{b\PYZhy{}}\PY{l+s+s1}{\PYZsq{}}\PY{p}{,} \PY{n}{label}\PY{o}{=}\PY{l+s+s1}{\PYZsq{}}\PY{l+s+s1}{theory}\PY{l+s+s1}{\PYZsq{}}\PY{p}{)}
          \PY{n}{plt}\PY{o}{.}\PY{n}{plot}\PY{p}{(}\PY{n}{xdata}\PY{p}{,} \PY{n}{ydata}\PY{p}{,} \PY{l+s+s1}{\PYZsq{}}\PY{l+s+s1}{ro}\PY{l+s+s1}{\PYZsq{}}\PY{p}{,} \PY{n}{label}\PY{o}{=}\PY{l+s+s2}{\PYZdq{}}\PY{l+s+s2}{data}\PY{l+s+s2}{\PYZdq{}}\PY{p}{)}
          \PY{n}{plt}\PY{o}{.}\PY{n}{xlabel}\PY{p}{(}\PY{l+s+s1}{\PYZsq{}}\PY{l+s+s1}{x}\PY{l+s+s1}{\PYZsq{}}\PY{p}{)}
          \PY{n}{plt}\PY{o}{.}\PY{n}{ylabel}\PY{p}{(}\PY{l+s+s1}{\PYZsq{}}\PY{l+s+s1}{transverse displacement}\PY{l+s+s1}{\PYZsq{}}\PY{p}{)}
          \PY{n}{plt}\PY{o}{.}\PY{n}{legend}\PY{p}{(}\PY{n}{loc}\PY{o}{=}\PY{l+s+s1}{\PYZsq{}}\PY{l+s+s1}{upper right}\PY{l+s+s1}{\PYZsq{}}\PY{p}{)}
          \PY{n}{plt}\PY{o}{.}\PY{n}{axhline}\PY{p}{(}\PY{n}{color} \PY{o}{=} \PY{l+s+s1}{\PYZsq{}}\PY{l+s+s1}{gray}\PY{l+s+s1}{\PYZsq{}}\PY{p}{,} \PY{n}{zorder}\PY{o}{=}\PY{o}{\PYZhy{}}\PY{l+m+mi}{1}\PY{p}{)}
          \PY{n}{plt}\PY{o}{.}\PY{n}{axvline}\PY{p}{(}\PY{n}{color} \PY{o}{=} \PY{l+s+s1}{\PYZsq{}}\PY{l+s+s1}{gray}\PY{l+s+s1}{\PYZsq{}}\PY{p}{,} \PY{n}{zorder}\PY{o}{=}\PY{o}{\PYZhy{}}\PY{l+m+mi}{1}\PY{p}{)}
          \PY{c+c1}{\PYZsh{} save plot to file}
          \PY{n}{plt}\PY{o}{.}\PY{n}{savefig}\PY{p}{(}\PY{l+s+s1}{\PYZsq{}}\PY{l+s+s1}{WavyPulse.pdf}\PY{l+s+s1}{\PYZsq{}}\PY{p}{)} \PY{c+c1}{\PYZsh{} display plot on screen}
\end{Verbatim}


    \begin{center}
    \adjustimage{max size={0.9\linewidth}{0.9\paperheight}}{output_87_0.png}
    \end{center}
    { \hspace*{\fill} \\}
    
    The functions that do the plotting begin on line 12. Let's go through
them one by one and see what they do. You will notice in several cases
that keyword arguments (kwargs) are used in several cases. Keyword
arguments are optional arguments that have the form kwarg= data, where
data might be a number, a string, a tuple, or some other form of data.

\begin{itemize}
\item
  \textbf{figure()} creates a blank figure window. If it has no
  arguments, it creates a window that is 8 inches wide and 6 inches high
  by default, although the size that appears on your computer depends on
  your screen's resolution. For most computers, it will be much smaller.
  You can create a window whose size differs from the default using the
  optional keyword argument figsize, as we have done here. If you use
  figsize, set it equal to a 2-element tuple where the elements are the
  width and height, respectively, of the plot. Multiple calls to
  figure() opens multiple windows: figure(1) opens up one window for
  plotting, figure(2) another, and figure(3) yet another.
\item
  \textbf{plot(x, y, optional arguments )} graphs the x-y data in the
  arrays x and y. The third argument is a format string that specifies
  the color and the type of line or symbol that is used to plot the
  data. The string 'ro' specifies a red (r) circle (o). The string 'b-'
  specifies a blue (b) solid line (-). The keyword argument label is set
  equal to a string that labels the data if the legend function is
  called subsequently.
\item
  \textbf{xlabel( string )} takes a string argument that specifies the
  label for the graph's x-axis.
\item
  \textbf{ylabel( string )} takes a string argument that specifies the
  label for the graph's y-axis.
\item
  \textbf{legend()} makes a legend for the data plotted. Each x-y data
  set is labeled using the string that was supplied by the label keyword
  in the plot function that graphed the data set. The loc keyword
  argument specifies the location of the legend.
\item
  \textbf{axhline()} draws a horizontal line across the width of the
  plot at y=0. The optional keyword argument color is a string that
  specifies the color of the line. The default color is black. The
  optional keyword argument zorder is an integer that specifies which
  plotting elements are in front of or behind others. By default, new
  plotting elements appear on top of previously plotted elements and
  have a value of zorder=0. By specifying zorder=-1, the horizontal line
  is plotted behind all ex- isting plot elements that have not be
  assigned an explicit zorder less than -1.
\item
  \textbf{axvline()} draws a vertical line from the top to the bottom of
  the plot at x=0. See axhline() for explanation of the arguments.
\item
  \textbf{savefig( string )} saves the figure to data data file with a
  name specified by the string argument. The string argument can also
  contain path information if you want to save the file so some place
  other than the default directory.
\item
  \textbf{show()} displays the plot on the computer screen. No screen
  output is produced before this function is called.
\end{itemize}

    \hypertarget{error-bars}{%
\subsubsection{Error bars}\label{error-bars}}

When plotting experimental data it is customary to include error bars
that indicate graphically the degree of uncertainty that exists in the
measurement of each data point. The MatPlotLib function errorbar plots
data with error bars attached. It can be used in a way that either
replaces or augments the plot function. Both vertical and horizontal
error bars can be displayed. The figure below illustrates the use of
error bars.

The code and plot below illustrates how to make error bars and was used
to make the above plot. Lines 14 and 15 contain the call to the errorbar
function. The x error bars are all set to a constant value of 0.75,
meaning that the error bars extend 0.75 to the left and 0.75 to the
right of each data point. The y error bars are set equal to an array,
which was read in from the data file containing the data to be plotted,
so each data point has a different y error bar. By the way, leaving out
the xerr keyword argument in the errorbar function call below would mean
that only the y error bars would be plotted.

    \begin{Verbatim}[commandchars=\\\{\}]
{\color{incolor}In [{\color{incolor}449}]:} \PY{k+kn}{import} \PY{n+nn}{numpy} \PY{k}{as} \PY{n+nn}{np}
          \PY{k+kn}{import} \PY{n+nn}{matplotlib}\PY{n+nn}{.}\PY{n+nn}{pyplot} \PY{k}{as} \PY{n+nn}{plt}
          \PY{c+c1}{\PYZsh{} read data from file}
          \PY{n}{dataPt}\PY{p}{,} \PY{n}{xdata}\PY{p}{,} \PY{n}{ydata}\PY{p}{,} \PY{n}{yerror} \PY{o}{=} \PY{n}{np}\PY{o}{.}\PY{n}{loadtxt}\PY{p}{(}\PY{l+s+s2}{\PYZdq{}}\PY{l+s+s2}{MyData.txt}\PY{l+s+s2}{\PYZdq{}}\PY{p}{,} \PY{n}{skiprows}\PY{o}{=}\PY{l+m+mi}{5} \PY{p}{,} \PY{n}{unpack}\PY{o}{=}\PY{k+kc}{True}\PY{p}{)}
          \PY{c+c1}{\PYZsh{} create plot}
          \PY{n}{plt}\PY{o}{.}\PY{n}{figure}\PY{p}{(}\PY{l+m+mi}{1}\PY{p}{,} \PY{n}{figsize} \PY{o}{=} \PY{p}{(}\PY{l+m+mi}{8}\PY{p}{,}\PY{l+m+mi}{6}\PY{p}{)} \PY{p}{)}
          \PY{n}{plt}\PY{o}{.}\PY{n}{errorbar}\PY{p}{(}\PY{n}{xdata}\PY{p}{,} \PY{n}{ydata}\PY{p}{,} \PY{n}{fmt}\PY{o}{=}\PY{l+s+s2}{\PYZdq{}}\PY{l+s+s2}{ro}\PY{l+s+s2}{\PYZdq{}}\PY{p}{,} \PY{n}{label}\PY{o}{=}\PY{l+s+s2}{\PYZdq{}}\PY{l+s+s2}{data}\PY{l+s+s2}{\PYZdq{}}\PY{p}{,}
                       \PY{n}{xerr}\PY{o}{=}\PY{l+m+mf}{0.15}\PY{p}{,} \PY{n}{yerr}\PY{o}{=}\PY{n}{yerror}\PY{p}{,} \PY{n}{ecolor}\PY{o}{=}\PY{l+s+s2}{\PYZdq{}}\PY{l+s+s2}{black}\PY{l+s+s2}{\PYZdq{}}\PY{p}{)}
          \PY{n}{plt}\PY{o}{.}\PY{n}{xlabel}\PY{p}{(}\PY{l+s+s2}{\PYZdq{}}\PY{l+s+s2}{x}\PY{l+s+s2}{\PYZdq{}}\PY{p}{)}
          \PY{n}{plt}\PY{o}{.}\PY{n}{ylabel}\PY{p}{(}\PY{l+s+s2}{\PYZdq{}}\PY{l+s+s2}{transverse displacement}\PY{l+s+s2}{\PYZdq{}}\PY{p}{)}
          \PY{n}{plt}\PY{o}{.}\PY{n}{legend}\PY{p}{(}\PY{n}{loc}\PY{o}{=}\PY{l+s+s2}{\PYZdq{}}\PY{l+s+s2}{upper right}\PY{l+s+s2}{\PYZdq{}}\PY{p}{)}
          \PY{c+c1}{\PYZsh{} save plot to file}
          \PY{n}{plt}\PY{o}{.}\PY{n}{savefig}\PY{p}{(}\PY{l+s+s2}{\PYZdq{}}\PY{l+s+s2}{ExpDecay.pdf}\PY{l+s+s2}{\PYZdq{}}\PY{p}{)}
          \PY{c+c1}{\PYZsh{} display plot on screen}
          \PY{n}{plt}\PY{o}{.}\PY{n}{show}\PY{p}{(}\PY{p}{)}
\end{Verbatim}


    \begin{center}
    \adjustimage{max size={0.9\linewidth}{0.9\paperheight}}{output_90_0.png}
    \end{center}
    { \hspace*{\fill} \\}
    
    \hypertarget{setting-plotting-limits-and-excluding-data}{%
\subsubsection{Setting plotting limits and excluding
data}\label{setting-plotting-limits-and-excluding-data}}

    \begin{Verbatim}[commandchars=\\\{\}]
{\color{incolor}In [{\color{incolor}450}]:} \PY{k+kn}{import} \PY{n+nn}{numpy} \PY{k}{as} \PY{n+nn}{np}
          \PY{k+kn}{import} \PY{n+nn}{matplotlib}\PY{n+nn}{.}\PY{n+nn}{pyplot} \PY{k}{as} \PY{n+nn}{plt}
          \PY{n}{theta} \PY{o}{=} \PY{n}{np}\PY{o}{.}\PY{n}{arange}\PY{p}{(}\PY{l+m+mf}{0.01}\PY{p}{,} \PY{l+m+mf}{10.}\PY{p}{,} \PY{l+m+mf}{0.04}\PY{p}{)}
          \PY{n}{ytan} \PY{o}{=} \PY{n}{np}\PY{o}{.}\PY{n}{tan}\PY{p}{(}\PY{n}{theta}\PY{p}{)}
          \PY{n}{plt}\PY{o}{.}\PY{n}{figure}\PY{p}{(}\PY{n}{figsize}\PY{o}{=}\PY{p}{(}\PY{l+m+mi}{8}\PY{p}{,}\PY{l+m+mi}{6}\PY{p}{)}\PY{p}{)}
          \PY{n}{plt}\PY{o}{.}\PY{n}{plot}\PY{p}{(}\PY{n}{theta}\PY{p}{,} \PY{n}{ytan}\PY{p}{)}
          \PY{n}{plt}\PY{o}{.}\PY{n}{ylim}\PY{p}{(}\PY{o}{\PYZhy{}}\PY{l+m+mi}{8}\PY{p}{,} \PY{l+m+mi}{8}\PY{p}{)} \PY{c+c1}{\PYZsh{} restricts range of y axis from \PYZhy{}8 to +8 plt.axhline(color=\PYZdq{}gray\PYZdq{}, zorder=\PYZhy{}1)}
          \PY{n}{plt}\PY{o}{.}\PY{n}{show}\PY{p}{(}\PY{p}{)}
\end{Verbatim}


    \begin{center}
    \adjustimage{max size={0.9\linewidth}{0.9\paperheight}}{output_92_0.png}
    \end{center}
    { \hspace*{\fill} \\}
    
    ** Masked arrays**

    \begin{Verbatim}[commandchars=\\\{\}]
{\color{incolor}In [{\color{incolor}451}]:} \PY{k+kn}{import} \PY{n+nn}{numpy} \PY{k}{as} \PY{n+nn}{np}
          \PY{k+kn}{import} \PY{n+nn}{matplotlib}\PY{n+nn}{.}\PY{n+nn}{pyplot} \PY{k}{as} \PY{n+nn}{plt}
          \PY{n}{theta} \PY{o}{=} \PY{n}{np}\PY{o}{.}\PY{n}{arange}\PY{p}{(}\PY{l+m+mf}{0.01}\PY{p}{,} \PY{l+m+mf}{10.}\PY{p}{,} \PY{l+m+mf}{0.04}\PY{p}{)}
          \PY{n}{ytan} \PY{o}{=} \PY{n}{np}\PY{o}{.}\PY{n}{tan}\PY{p}{(}\PY{n}{theta}\PY{p}{)}
          \PY{n}{ytanM} \PY{o}{=} \PY{n}{np}\PY{o}{.}\PY{n}{ma}\PY{o}{.}\PY{n}{masked\PYZus{}where}\PY{p}{(}\PY{n}{np}\PY{o}{.}\PY{n}{abs}\PY{p}{(}\PY{n}{ytan}\PY{p}{)}\PY{o}{\PYZgt{}}\PY{l+m+mf}{20.}\PY{p}{,} \PY{n}{ytan}\PY{p}{)}
          \PY{n}{plt}\PY{o}{.}\PY{n}{figure}\PY{p}{(}\PY{n}{figsize}\PY{o}{=}\PY{p}{(}\PY{l+m+mi}{8}\PY{p}{,}\PY{l+m+mi}{6}\PY{p}{)}\PY{p}{)}
          \PY{n}{plt}\PY{o}{.}\PY{n}{plot}\PY{p}{(}\PY{n}{theta}\PY{p}{,} \PY{n}{ytanM}\PY{p}{)}
          \PY{n}{plt}\PY{o}{.}\PY{n}{ylim}\PY{p}{(}\PY{o}{\PYZhy{}}\PY{l+m+mi}{8}\PY{p}{,} \PY{l+m+mi}{8}\PY{p}{)}
          \PY{n}{plt}\PY{o}{.}\PY{n}{axhline}\PY{p}{(}\PY{n}{color}\PY{o}{=}\PY{l+s+s2}{\PYZdq{}}\PY{l+s+s2}{gray}\PY{l+s+s2}{\PYZdq{}}\PY{p}{,} \PY{n}{zorder}\PY{o}{=}\PY{o}{\PYZhy{}}\PY{l+m+mi}{1}\PY{p}{)}
          \PY{n}{plt}\PY{o}{.}\PY{n}{show}\PY{p}{(}\PY{p}{)}
\end{Verbatim}


    \begin{center}
    \adjustimage{max size={0.9\linewidth}{0.9\paperheight}}{output_94_0.png}
    \end{center}
    { \hspace*{\fill} \\}
    
    \hypertarget{subplots}{%
\subsubsection{Subplots}\label{subplots}}

Often you want to create two or more graphs and place them next to one
another, generally because they are related to each other in some way.

The function subplot, called on lines 13 and 24, creates the two
subplots in the above figure. subplot has three arguments. The first
specifies the number of rows that the figure space is to be divided
into; on line 13, it's two. The second specifies the number of columns
that the figure space is to be divided into; on line 13, it's one. The
third argument specifies which rectangle the will contain the plot
specified by the following function calls. Line 13 specifies that the
plotting commands that follow will be act on the first box. Line 24
specifies that the plotting commands that follow will be act on the
second box.

    \begin{Verbatim}[commandchars=\\\{\}]
{\color{incolor}In [{\color{incolor}452}]:} \PY{k+kn}{import} \PY{n+nn}{numpy} \PY{k}{as} \PY{n+nn}{np}
          \PY{k+kn}{import} \PY{n+nn}{matplotlib}\PY{n+nn}{.}\PY{n+nn}{pyplot} \PY{k}{as} \PY{n+nn}{plt}
          \PY{n}{theta} \PY{o}{=} \PY{n}{np}\PY{o}{.}\PY{n}{arange}\PY{p}{(}\PY{l+m+mf}{0.01}\PY{p}{,} \PY{l+m+mf}{8.}\PY{p}{,} \PY{l+m+mf}{0.04}\PY{p}{)}
          \PY{n}{y} \PY{o}{=} \PY{n}{np}\PY{o}{.}\PY{n}{sqrt}\PY{p}{(}\PY{p}{(}\PY{l+m+mf}{8.}\PY{o}{/}\PY{n}{theta}\PY{p}{)}\PY{o}{*}\PY{o}{*}\PY{l+m+mi}{2}\PY{o}{\PYZhy{}}\PY{l+m+mf}{1.}\PY{p}{)}
          \PY{n}{ytan} \PY{o}{=} \PY{n}{np}\PY{o}{.}\PY{n}{tan}\PY{p}{(}\PY{n}{theta}\PY{p}{)}
          \PY{n}{ytan} \PY{o}{=} \PY{n}{np}\PY{o}{.}\PY{n}{ma}\PY{o}{.}\PY{n}{masked\PYZus{}where}\PY{p}{(}\PY{n}{np}\PY{o}{.}\PY{n}{abs}\PY{p}{(}\PY{n}{ytan}\PY{p}{)}\PY{o}{\PYZgt{}}\PY{l+m+mf}{20.}\PY{p}{,} \PY{n}{ytan}\PY{p}{)}
          \PY{n}{ycot} \PY{o}{=} \PY{l+m+mf}{1.}\PY{o}{/}\PY{n}{np}\PY{o}{.}\PY{n}{tan}\PY{p}{(}\PY{n}{theta}\PY{p}{)}
          
          \PY{n}{ycot} \PY{o}{=} \PY{n}{np}\PY{o}{.}\PY{n}{ma}\PY{o}{.}\PY{n}{masked\PYZus{}where}\PY{p}{(}\PY{n}{np}\PY{o}{.}\PY{n}{abs}\PY{p}{(}\PY{n}{ycot}\PY{p}{)}\PY{o}{\PYZgt{}}\PY{l+m+mf}{20.}\PY{p}{,} \PY{n}{ycot}\PY{p}{)}
          \PY{n}{plt}\PY{o}{.}\PY{n}{figure}\PY{p}{(}\PY{l+m+mi}{1}\PY{p}{,}\PY{n}{figsize}\PY{p}{(}\PY{l+m+mi}{8}\PY{p}{,}\PY{l+m+mi}{8}\PY{p}{)}\PY{p}{)}
          \PY{n}{plt}\PY{o}{.}\PY{n}{subplot}\PY{p}{(}\PY{l+m+mi}{2}\PY{p}{,} \PY{l+m+mi}{1}\PY{p}{,} \PY{l+m+mi}{1}\PY{p}{)}
          \PY{n}{plt}\PY{o}{.}\PY{n}{plot}\PY{p}{(}\PY{n}{theta}\PY{p}{,} \PY{n}{y}\PY{p}{)}
          \PY{n}{plt}\PY{o}{.}\PY{n}{plot}\PY{p}{(}\PY{n}{theta}\PY{p}{,} \PY{n}{ytan}\PY{p}{)}
          \PY{n}{plt}\PY{o}{.}\PY{n}{ylim}\PY{p}{(}\PY{o}{\PYZhy{}}\PY{l+m+mi}{8}\PY{p}{,} \PY{l+m+mi}{8}\PY{p}{)}
          \PY{n}{plt}\PY{o}{.}\PY{n}{axhline}\PY{p}{(}\PY{n}{color}\PY{o}{=}\PY{l+s+s2}{\PYZdq{}}\PY{l+s+s2}{gray}\PY{l+s+s2}{\PYZdq{}}\PY{p}{,} \PY{n}{zorder}\PY{o}{=}\PY{o}{\PYZhy{}}\PY{l+m+mi}{1}\PY{p}{)}
          \PY{n}{plt}\PY{o}{.}\PY{n}{axvline}\PY{p}{(}\PY{n}{x}\PY{o}{=}\PY{n}{np}\PY{o}{.}\PY{n}{pi}\PY{o}{/}\PY{l+m+mf}{2.}\PY{p}{,} \PY{n}{color}\PY{o}{=}\PY{l+s+s2}{\PYZdq{}}\PY{l+s+s2}{gray}\PY{l+s+s2}{\PYZdq{}}\PY{p}{,} \PY{n}{linestyle}\PY{o}{=}\PY{l+s+s2}{\PYZdq{}}\PY{l+s+s2}{\PYZhy{}\PYZhy{}}\PY{l+s+s2}{\PYZdq{}}\PY{p}{,} \PY{n}{zorder}\PY{o}{=}\PY{o}{\PYZhy{}}\PY{l+m+mi}{1}\PY{p}{)}
          \PY{n}{plt}\PY{o}{.}\PY{n}{axvline}\PY{p}{(}\PY{n}{x}\PY{o}{=}\PY{l+m+mf}{3.}\PY{o}{*}\PY{n}{np}\PY{o}{.}\PY{n}{pi}\PY{o}{/}\PY{l+m+mf}{2.}\PY{p}{,} \PY{n}{color}\PY{o}{=}\PY{l+s+s2}{\PYZdq{}}\PY{l+s+s2}{gray}\PY{l+s+s2}{\PYZdq{}}\PY{p}{,} \PY{n}{linestyle}\PY{o}{=}\PY{l+s+s2}{\PYZdq{}}\PY{l+s+s2}{\PYZhy{}\PYZhy{}}\PY{l+s+s2}{\PYZdq{}}\PY{p}{,} \PY{n}{zorder}\PY{o}{=}\PY{o}{\PYZhy{}}\PY{l+m+mi}{1}\PY{p}{)}
          \PY{n}{plt}\PY{o}{.}\PY{n}{axvline}\PY{p}{(}\PY{n}{x}\PY{o}{=}\PY{l+m+mf}{5.}\PY{o}{*}\PY{n}{np}\PY{o}{.}\PY{n}{pi}\PY{o}{/}\PY{l+m+mf}{2.}\PY{p}{,} \PY{n}{color}\PY{o}{=}\PY{l+s+s2}{\PYZdq{}}\PY{l+s+s2}{gray}\PY{l+s+s2}{\PYZdq{}}\PY{p}{,} \PY{n}{linestyle}\PY{o}{=}\PY{l+s+s2}{\PYZdq{}}\PY{l+s+s2}{\PYZhy{}\PYZhy{}}\PY{l+s+s2}{\PYZdq{}}\PY{p}{,} \PY{n}{zorder}\PY{o}{=}\PY{o}{\PYZhy{}}\PY{l+m+mi}{1}\PY{p}{)}
          \PY{n}{plt}\PY{o}{.}\PY{n}{xlabel}\PY{p}{(}\PY{l+s+s2}{\PYZdq{}}\PY{l+s+s2}{theta}\PY{l+s+s2}{\PYZdq{}}\PY{p}{)}
          \PY{n}{plt}\PY{o}{.}\PY{n}{ylabel}\PY{p}{(}\PY{l+s+s2}{\PYZdq{}}\PY{l+s+s2}{tan(theta)}\PY{l+s+s2}{\PYZdq{}}\PY{p}{)}
          
          \PY{n}{plt}\PY{o}{.}\PY{n}{subplot}\PY{p}{(}\PY{l+m+mi}{2}\PY{p}{,} \PY{l+m+mi}{1}\PY{p}{,} \PY{l+m+mi}{2}\PY{p}{)}
          \PY{n}{plt}\PY{o}{.}\PY{n}{plot}\PY{p}{(}\PY{n}{theta}\PY{p}{,} \PY{o}{\PYZhy{}}\PY{n}{y}\PY{p}{)}
          \PY{n}{plt}\PY{o}{.}\PY{n}{plot}\PY{p}{(}\PY{n}{theta}\PY{p}{,} \PY{n}{ycot}\PY{p}{)}
          \PY{n}{plt}\PY{o}{.}\PY{n}{ylim}\PY{p}{(}\PY{o}{\PYZhy{}}\PY{l+m+mi}{8}\PY{p}{,} \PY{l+m+mi}{8}\PY{p}{)}
          \PY{n}{plt}\PY{o}{.}\PY{n}{axhline}\PY{p}{(}\PY{n}{color}\PY{o}{=}\PY{l+s+s2}{\PYZdq{}}\PY{l+s+s2}{gray}\PY{l+s+s2}{\PYZdq{}}\PY{p}{,} \PY{n}{zorder}\PY{o}{=}\PY{o}{\PYZhy{}}\PY{l+m+mi}{1}\PY{p}{)}
          \PY{n}{plt}\PY{o}{.}\PY{n}{axvline}\PY{p}{(}\PY{n}{x}\PY{o}{=}\PY{n}{np}\PY{o}{.}\PY{n}{pi}\PY{p}{,} \PY{n}{color}\PY{o}{=}\PY{l+s+s2}{\PYZdq{}}\PY{l+s+s2}{gray}\PY{l+s+s2}{\PYZdq{}}\PY{p}{,} \PY{n}{linestyle}\PY{o}{=}\PY{l+s+s2}{\PYZdq{}}\PY{l+s+s2}{\PYZhy{}\PYZhy{}}\PY{l+s+s2}{\PYZdq{}}\PY{p}{,} \PY{n}{zorder}\PY{o}{=}\PY{o}{\PYZhy{}}\PY{l+m+mi}{1}\PY{p}{)}
          \PY{n}{plt}\PY{o}{.}\PY{n}{axvline}\PY{p}{(}\PY{n}{x}\PY{o}{=}\PY{l+m+mf}{2.}\PY{o}{*}\PY{n}{np}\PY{o}{.}\PY{n}{pi}\PY{p}{,} \PY{n}{color}\PY{o}{=}\PY{l+s+s2}{\PYZdq{}}\PY{l+s+s2}{gray}\PY{l+s+s2}{\PYZdq{}}\PY{p}{,} \PY{n}{linestyle}\PY{o}{=}\PY{l+s+s2}{\PYZdq{}}\PY{l+s+s2}{\PYZhy{}\PYZhy{}}\PY{l+s+s2}{\PYZdq{}}\PY{p}{,} \PY{n}{zorder}\PY{o}{=}\PY{o}{\PYZhy{}}\PY{l+m+mi}{1}\PY{p}{)}
          \PY{n}{plt}\PY{o}{.}\PY{n}{xlabel}\PY{p}{(}\PY{l+s+sa}{r}\PY{l+s+s1}{\PYZsq{}}\PY{l+s+s1}{\PYZdl{}}\PY{l+s+s1}{\PYZbs{}}\PY{l+s+s1}{theta\PYZdl{}}\PY{l+s+s1}{\PYZsq{}}\PY{p}{)}
          \PY{n}{plt}\PY{o}{.}\PY{n}{ylabel}\PY{p}{(}\PY{l+s+sa}{r}\PY{l+s+s1}{\PYZsq{}}\PY{l+s+s1}{cot(\PYZdl{}}\PY{l+s+s1}{\PYZbs{}}\PY{l+s+s1}{theta\PYZdl{})}\PY{l+s+s1}{\PYZsq{}}\PY{p}{)}
          
          \PY{n}{plt}\PY{o}{.}\PY{n}{tight\PYZus{}layout}\PY{p}{(}\PY{p}{)}
          \PY{n}{plt}\PY{o}{.}\PY{n}{show}\PY{p}{(}\PY{p}{)}
\end{Verbatim}


    \begin{center}
    \adjustimage{max size={0.9\linewidth}{0.9\paperheight}}{output_96_0.png}
    \end{center}
    { \hspace*{\fill} \\}
    
    \begin{center}\rule{0.5\linewidth}{\linethickness}\end{center}

\hypertarget{logarithmic-plots}{%
\subsection{Logarithmic plots}\label{logarithmic-plots}}

Data sets can span many orders of magnitude from fractional quantities
much smaller than unity to values much larger than unity. In such cases
it is often useful to plot the data on logarithmic axes.

    \hypertarget{semi-log-plots}{%
\subsubsection{Semi-log plots}\label{semi-log-plots}}

For data sets that vary exponentially in the independent variable, it is
often useful to use one or more logarithmic axes. Radioactive decay of
unstable nuclei, for example, exhibits an exponential decrease in the
number of particles emitted from the nuclei as a function of time.

MatPlotLib provides two functions for making semi-logarithmic plots,
semilogx and semilogy, for creating plots with logarithmic x and y axes,
with linear y and x axes, respectively. We illustrate their use in the
program below, which made the above plots.

    \begin{Verbatim}[commandchars=\\\{\}]
{\color{incolor}In [{\color{incolor}2}]:} \PY{k+kn}{import} \PY{n+nn}{numpy} \PY{k}{as} \PY{n+nn}{np}
        \PY{k+kn}{import} \PY{n+nn}{matplotlib}\PY{n+nn}{.}\PY{n+nn}{pyplot} \PY{k}{as} \PY{n+nn}{plt}
        \PY{o}{\PYZpc{}}\PY{k}{config} InlineBackend.figure\PYZus{}format = \PYZsq{}retina\PYZsq{}
        
        \PY{c+c1}{\PYZsh{} read data from file}
        \PY{n}{time}\PY{p}{,} \PY{n}{counts}\PY{p}{,} \PY{n}{unc} \PY{o}{=} \PY{n}{np}\PY{o}{.}\PY{n}{loadtxt}\PY{p}{(}\PY{l+s+s1}{\PYZsq{}}\PY{l+s+s1}{SemilogDemo.txt}\PY{l+s+s1}{\PYZsq{}}\PY{p}{,} \PY{n}{unpack}\PY{o}{=}\PY{k+kc}{True}\PY{p}{)}
        
        \PY{c+c1}{\PYZsh{} create theoretical fitting curve}
        \PY{n}{tau} \PY{o}{=} \PY{l+m+mf}{20.2} \PY{c+c1}{\PYZsh{} Phosphorus\PYZhy{}32 half life = 14 days; tau = t\PYZus{}half/ln(2)}
        \PY{n}{N0} \PY{o}{=} \PY{l+m+mf}{8200.} \PY{c+c1}{\PYZsh{} Initial count rate (per second)}
        \PY{n}{t} \PY{o}{=} \PY{n}{np}\PY{o}{.}\PY{n}{linspace}\PY{p}{(}\PY{l+m+mi}{0}\PY{p}{,} \PY{l+m+mi}{180}\PY{p}{,} \PY{l+m+mi}{128}\PY{p}{)}
        \PY{n}{N} \PY{o}{=} \PY{n}{N0} \PY{o}{*} \PY{n}{np}\PY{o}{.}\PY{n}{exp}\PY{p}{(}\PY{o}{\PYZhy{}}\PY{n}{t}\PY{o}{/}\PY{n}{tau}\PY{p}{)}
        
        \PY{c+c1}{\PYZsh{} create plot}
        \PY{n}{plt}\PY{o}{.}\PY{n}{figure}\PY{p}{(}\PY{l+m+mi}{1}\PY{p}{,} \PY{n}{figsize} \PY{o}{=} \PY{p}{(}\PY{l+m+mi}{10}\PY{p}{,}\PY{l+m+mi}{4}\PY{p}{)} \PY{p}{)}
        
        \PY{n}{plt}\PY{o}{.}\PY{n}{subplot}\PY{p}{(}\PY{l+m+mi}{1}\PY{p}{,} \PY{l+m+mi}{2}\PY{p}{,} \PY{l+m+mi}{1}\PY{p}{)}
        \PY{n}{plt}\PY{o}{.}\PY{n}{plot}\PY{p}{(}\PY{n}{t}\PY{p}{,} \PY{n}{N}\PY{p}{,} \PY{l+s+s1}{\PYZsq{}}\PY{l+s+s1}{b\PYZhy{}}\PY{l+s+s1}{\PYZsq{}}\PY{p}{,} \PY{n}{label}\PY{o}{=}\PY{l+s+s2}{\PYZdq{}}\PY{l+s+s2}{theory}\PY{l+s+s2}{\PYZdq{}}\PY{p}{)}
        \PY{n}{plt}\PY{o}{.}\PY{n}{plot}\PY{p}{(}\PY{n}{time}\PY{p}{,} \PY{n}{counts}\PY{p}{,} \PY{l+s+s1}{\PYZsq{}}\PY{l+s+s1}{ro}\PY{l+s+s1}{\PYZsq{}}\PY{p}{,} \PY{n}{label}\PY{o}{=}\PY{l+s+s2}{\PYZdq{}}\PY{l+s+s2}{data}\PY{l+s+s2}{\PYZdq{}}\PY{p}{)}
        \PY{n}{plt}\PY{o}{.}\PY{n}{xlabel}\PY{p}{(}\PY{l+s+s1}{\PYZsq{}}\PY{l+s+s1}{time (days)}\PY{l+s+s1}{\PYZsq{}}\PY{p}{)}
        \PY{n}{plt}\PY{o}{.}\PY{n}{ylabel}\PY{p}{(}\PY{l+s+s1}{\PYZsq{}}\PY{l+s+s1}{counts per second}\PY{l+s+s1}{\PYZsq{}}\PY{p}{)}
        \PY{n}{plt}\PY{o}{.}\PY{n}{legend}\PY{p}{(}\PY{n}{loc}\PY{o}{=}\PY{l+s+s1}{\PYZsq{}}\PY{l+s+s1}{upper right}\PY{l+s+s1}{\PYZsq{}}\PY{p}{)}
        
        \PY{n}{plt}\PY{o}{.}\PY{n}{subplot}\PY{p}{(}\PY{l+m+mi}{1}\PY{p}{,} \PY{l+m+mi}{2}\PY{p}{,} \PY{l+m+mi}{2}\PY{p}{)}
        \PY{n}{plt}\PY{o}{.}\PY{n}{semilogy}\PY{p}{(}\PY{n}{t}\PY{p}{,} \PY{n}{N}\PY{p}{,} \PY{l+s+s1}{\PYZsq{}}\PY{l+s+s1}{b\PYZhy{}}\PY{l+s+s1}{\PYZsq{}}\PY{p}{,} \PY{n}{label}\PY{o}{=}\PY{l+s+s2}{\PYZdq{}}\PY{l+s+s2}{theory}\PY{l+s+s2}{\PYZdq{}}\PY{p}{)}
        \PY{n}{plt}\PY{o}{.}\PY{n}{semilogy}\PY{p}{(}\PY{n}{time}\PY{p}{,} \PY{n}{counts}\PY{p}{,} \PY{l+s+s1}{\PYZsq{}}\PY{l+s+s1}{ro}\PY{l+s+s1}{\PYZsq{}}\PY{p}{,} \PY{n}{label}\PY{o}{=}\PY{l+s+s2}{\PYZdq{}}\PY{l+s+s2}{data}\PY{l+s+s2}{\PYZdq{}}\PY{p}{)}
        \PY{n}{plt}\PY{o}{.}\PY{n}{xlabel}\PY{p}{(}\PY{l+s+s1}{\PYZsq{}}\PY{l+s+s1}{time (days)}\PY{l+s+s1}{\PYZsq{}}\PY{p}{)}
        \PY{n}{plt}\PY{o}{.}\PY{n}{ylabel}\PY{p}{(}\PY{l+s+s1}{\PYZsq{}}\PY{l+s+s1}{counts per second}\PY{l+s+s1}{\PYZsq{}}\PY{p}{)}
        \PY{n}{plt}\PY{o}{.}\PY{n}{legend}\PY{p}{(}\PY{n}{loc}\PY{o}{=}\PY{l+s+s1}{\PYZsq{}}\PY{l+s+s1}{upper right}\PY{l+s+s1}{\PYZsq{}}\PY{p}{)}
        
        \PY{n}{plt}\PY{o}{.}\PY{n}{tight\PYZus{}layout}\PY{p}{(}\PY{p}{)}
        \PY{n}{plt}\PY{o}{.}\PY{n}{show}\PY{p}{(}\PY{p}{)}
\end{Verbatim}


    \begin{center}
    \adjustimage{max size={0.9\linewidth}{0.9\paperheight}}{output_99_0.png}
    \end{center}
    { \hspace*{\fill} \\}
    
    \hypertarget{log-log-plots}{%
\subsubsection{Log-log plots}\label{log-log-plots}}

MatPlotLib can also make log-log or double-logarithmic plots using the
function loglog. It is useful when both the \(x\) and \(y\) data span
many orders of magnitude. Data that are described by a power law
\(y=Ax^b\), where \(A\) and \(b\) are constants, appear as straight
lines when plotted on a log-log plot. Again, the loglog function works
just like the plot function but with logarithmic axes.

    \textbf{Exercise}

Create a data file with the data shown below.

\begin{itemize}
\item
  Read the data into Python program and plot \(t\) vs \(y\) using
  circles for data points with error bars. Use the data in the \(dy\)
  column as the error estimates for the \(y\) data. Label the horizontal
  and vertical axes ``time (s)'' and ``position (cm)''.
\item
  On the same graph, plot the function below as a smooth line. Make the
  line pass behind the data points.
\end{itemize}

\begin{equation}
y(t)=\left [3+\frac{1}{2}\sin\frac{\pi t}{5} \right]t e^{-t/10}
\end{equation}

\begin{verbatim}
Data for Exercise
Date: 8-May-2017
Data taken by Karl and Heinz

 t      d       dy
 1.0    2.94    0.7
 4.5    8.29    1.2
 8.0    9.36    1.2
11.5   11.60    1.4
15.0    9.32    1.3
18.5    7.75    1.1
22.0    8.06    1.2
25.5    5.60    1.0
29.0    4.50    0.8
32.5    4.01    0.8
36.0    2.62    0.7
39.5    1.70    0.6
43.0    2.03    0.6
\end{verbatim}

    \begin{Verbatim}[commandchars=\\\{\}]
{\color{incolor}In [{\color{incolor} }]:} \PY{c+c1}{\PYZsh{} code goes here}
\end{Verbatim}


    \begin{center}\rule{0.5\linewidth}{\linethickness}\end{center}

\hypertarget{contour-and-density-plots}{%
\subsection{Contour and Density Plots}\label{contour-and-density-plots}}

A contour plots are useful tools to study two dimensional data, meaning
Z(X,Y). We use a simple example of a wave interference for two
dimensional plotting in contour and density plots.

Our wave shall be represented by a snapshot of a spherical wave at a
certain time.

\begin{equation}
Z(X,Y)=\frac{1}{r}\sin\left( \frac{2\pi}{\lambda} r \right)
\end{equation}

Here \(r\) is the distance from the source \(r=\sqrt{X^2+Y^2}\). To show
interference, we just use two of those waves as a superposition. To keep
it simple we will skip the \(1/r\) amplitude decay and just use the
\(\sin\) part. We don't care about the details of the physics here. The
code might be useful for later studies.

    \begin{Verbatim}[commandchars=\\\{\}]
{\color{incolor}In [{\color{incolor}563}]:} \PY{n}{lmda} \PY{o}{=} \PY{l+m+mi}{2} \PY{c+c1}{\PYZsh{} defines the wavelength }
          \PY{n}{x01}\PY{o}{=}\PY{l+m+mf}{1.5}\PY{o}{*}\PY{n}{np}\PY{o}{.}\PY{n}{pi} \PY{c+c1}{\PYZsh{} location of the first source, y01=0}
          \PY{n}{x02}\PY{o}{=}\PY{l+m+mf}{2.5}\PY{o}{*}\PY{n}{np}\PY{o}{.}\PY{n}{pi} \PY{c+c1}{\PYZsh{} location of the second source, y02=0}
          \PY{n}{x} \PY{o}{=} \PY{n}{np}\PY{o}{.}\PY{n}{linspace}\PY{p}{(}\PY{l+m+mi}{0}\PY{p}{,} \PY{l+m+mi}{4}\PY{o}{*}\PY{n}{np}\PY{o}{.}\PY{n}{pi}\PY{p}{,} \PY{l+m+mi}{100}\PY{p}{)}
          \PY{n}{y} \PY{o}{=} \PY{n}{np}\PY{o}{.}\PY{n}{linspace}\PY{p}{(}\PY{l+m+mi}{0}\PY{p}{,} \PY{l+m+mi}{4}\PY{o}{*}\PY{n}{np}\PY{o}{.}\PY{n}{pi}\PY{p}{,} \PY{l+m+mi}{100}\PY{p}{)}
          
          
          \PY{n}{X}\PY{p}{,} \PY{n}{Y} \PY{o}{=} \PY{n}{np}\PY{o}{.}\PY{n}{meshgrid}\PY{p}{(}\PY{n}{x}\PY{p}{,} \PY{n}{y}\PY{p}{)}
          \PY{n}{Z} \PY{o}{=}  \PY{p}{(}\PY{n}{np}\PY{o}{.}\PY{n}{sin}\PY{p}{(}\PY{n}{np}\PY{o}{.}\PY{n}{sqrt}\PY{p}{(}\PY{p}{(}\PY{n}{X}\PY{o}{\PYZhy{}}\PY{n}{x01}\PY{p}{)}\PY{o}{*}\PY{o}{*}\PY{l+m+mi}{2}\PY{o}{+}\PY{n}{Y}\PY{o}{*}\PY{o}{*}\PY{l+m+mi}{2}\PY{p}{)}\PY{o}{*}\PY{l+m+mi}{2}\PY{o}{*}\PY{n}{np}\PY{o}{.}\PY{n}{pi}\PY{o}{/}\PY{n}{lmda}\PY{p}{)}\PY{o}{+}\PY{n}{np}\PY{o}{.}\PY{n}{sin}\PY{p}{(}\PY{n}{np}\PY{o}{.}\PY{n}{sqrt}\PY{p}{(}\PY{p}{(}\PY{n}{X}\PY{o}{\PYZhy{}}\PY{n}{x02}\PY{p}{)}\PY{o}{*}\PY{o}{*}\PY{l+m+mi}{2}\PY{o}{+}\PY{n}{Y}\PY{o}{*}\PY{o}{*}\PY{l+m+mi}{2}\PY{p}{)}\PY{o}{*}\PY{l+m+mi}{2}\PY{o}{*}\PY{n}{np}\PY{o}{.}\PY{n}{pi}\PY{o}{/}\PY{n}{lmda}\PY{p}{)}\PY{p}{)}\PY{o}{*}\PY{o}{*}\PY{l+m+mi}{2}
\end{Verbatim}


    \begin{Verbatim}[commandchars=\\\{\}]
{\color{incolor}In [{\color{incolor}564}]:} \PY{n}{plt}\PY{o}{.}\PY{n}{figure}\PY{p}{(}\PY{l+m+mi}{1}\PY{p}{,}\PY{n}{figsize}\PY{p}{(}\PY{l+m+mi}{6}\PY{p}{,}\PY{l+m+mi}{6}\PY{p}{)}\PY{p}{)}
          \PY{n}{plt}\PY{o}{.}\PY{n}{contour}\PY{p}{(}\PY{n}{X}\PY{p}{,} \PY{n}{Y}\PY{p}{,} \PY{n}{Z}\PY{p}{,} \PY{l+m+mi}{2}\PY{p}{,} \PY{n}{colors}\PY{o}{=}\PY{l+s+s1}{\PYZsq{}}\PY{l+s+s1}{black}\PY{l+s+s1}{\PYZsq{}}\PY{p}{)}\PY{p}{;}
\end{Verbatim}


    \begin{center}
    \adjustimage{max size={0.9\linewidth}{0.9\paperheight}}{output_105_0.png}
    \end{center}
    { \hspace*{\fill} \\}
    
    \begin{Verbatim}[commandchars=\\\{\}]
{\color{incolor}In [{\color{incolor}565}]:} \PY{n}{plt}\PY{o}{.}\PY{n}{figure}\PY{p}{(}\PY{l+m+mi}{1}\PY{p}{,}\PY{n}{figsize}\PY{p}{(}\PY{l+m+mi}{8}\PY{p}{,}\PY{l+m+mi}{7}\PY{p}{)}\PY{p}{)}
          \PY{n}{plt}\PY{o}{.}\PY{n}{contourf}\PY{p}{(}\PY{n}{X}\PY{p}{,} \PY{n}{Y}\PY{p}{,} \PY{n}{Z}\PY{p}{,} \PY{l+m+mi}{10}\PY{p}{,} \PY{n}{cmap}\PY{o}{=}\PY{l+s+s1}{\PYZsq{}}\PY{l+s+s1}{RdGy}\PY{l+s+s1}{\PYZsq{}}\PY{p}{)}
          \PY{n}{plt}\PY{o}{.}\PY{n}{colorbar}\PY{p}{(}\PY{p}{)}\PY{p}{;}
\end{Verbatim}


    \begin{center}
    \adjustimage{max size={0.9\linewidth}{0.9\paperheight}}{output_106_0.png}
    \end{center}
    { \hspace*{\fill} \\}
    
    \begin{Verbatim}[commandchars=\\\{\}]
{\color{incolor}In [{\color{incolor}566}]:} \PY{n}{plt}\PY{o}{.}\PY{n}{figure}\PY{p}{(}\PY{l+m+mi}{1}\PY{p}{,}\PY{n}{figsize}\PY{p}{(}\PY{l+m+mi}{8}\PY{p}{,}\PY{l+m+mi}{7}\PY{p}{)}\PY{p}{)}
          \PY{n}{plt}\PY{o}{.}\PY{n}{imshow}\PY{p}{(}\PY{n}{Z}\PY{p}{)}\PY{p}{;}
          \PY{n}{plt}\PY{o}{.}\PY{n}{colorbar}\PY{p}{(}\PY{p}{)}\PY{p}{;}
\end{Verbatim}


    \begin{center}
    \adjustimage{max size={0.9\linewidth}{0.9\paperheight}}{output_107_0.png}
    \end{center}
    { \hspace*{\fill} \\}
    
    \hypertarget{d-plotting}{%
\subsubsection{3D Plotting}\label{d-plotting}}

Matplotlib was initially designed with only two-dimensional plotting in
mind. Around the time of the 1.0 release, some three-dimensional
plotting utilities were built on top of Matplotlib's two-dimensional
display, and the result is a convenient (if somewhat limited) set of
tools for three-dimensional data visualization. Three-dimensional plots
are enabled by importing the mplot3d toolkit, included with the main
Matplotlib installation:

    \begin{Verbatim}[commandchars=\\\{\}]
{\color{incolor}In [{\color{incolor}1}]:} \PY{k+kn}{from} \PY{n+nn}{mpl\PYZus{}toolkits} \PY{k}{import} \PY{n}{mplot3d}
\end{Verbatim}


    Once this submodule is imported, a three-dimensional axes can be created
by passing the keyword projection=`3d' to any of the normal axes
creation routines:

    \begin{Verbatim}[commandchars=\\\{\}]
{\color{incolor}In [{\color{incolor}2}]:} \PY{k+kn}{import} \PY{n+nn}{numpy} \PY{k}{as} \PY{n+nn}{np}
        \PY{k+kn}{import} \PY{n+nn}{matplotlib}\PY{n+nn}{.}\PY{n+nn}{pyplot} \PY{k}{as} \PY{n+nn}{plt}
\end{Verbatim}


    \begin{Verbatim}[commandchars=\\\{\}]
{\color{incolor}In [{\color{incolor}3}]:} \PY{n}{plt}\PY{o}{.}\PY{n}{figure}\PY{p}{(}\PY{n}{figsize}\PY{o}{=}\PY{p}{(}\PY{l+m+mi}{10}\PY{p}{,}\PY{l+m+mi}{6}\PY{p}{)}\PY{p}{)}
        \PY{n}{ax} \PY{o}{=} \PY{n}{plt}\PY{o}{.}\PY{n}{axes}\PY{p}{(}\PY{n}{projection}\PY{o}{=}\PY{l+s+s1}{\PYZsq{}}\PY{l+s+s1}{3d}\PY{l+s+s1}{\PYZsq{}}\PY{p}{)}
        \PY{n}{ax}\PY{o}{.}\PY{n}{set\PYZus{}xlabel}\PY{p}{(}\PY{l+s+s1}{\PYZsq{}}\PY{l+s+s1}{x}\PY{l+s+s1}{\PYZsq{}}\PY{p}{)}
        \PY{n}{ax}\PY{o}{.}\PY{n}{set\PYZus{}ylabel}\PY{p}{(}\PY{l+s+s1}{\PYZsq{}}\PY{l+s+s1}{y}\PY{l+s+s1}{\PYZsq{}}\PY{p}{)}
        \PY{n}{ax}\PY{o}{.}\PY{n}{set\PYZus{}zlabel}\PY{p}{(}\PY{l+s+s1}{\PYZsq{}}\PY{l+s+s1}{z}\PY{l+s+s1}{\PYZsq{}}\PY{p}{)}
\end{Verbatim}


\begin{Verbatim}[commandchars=\\\{\}]
{\color{outcolor}Out[{\color{outcolor}3}]:} Text(0.5,0,'z')
\end{Verbatim}
            
    \begin{center}
    \adjustimage{max size={0.9\linewidth}{0.9\paperheight}}{output_112_1.png}
    \end{center}
    { \hspace*{\fill} \\}
    
    With this three-dimensional axes enabled, we can now plot a variety of
three-dimensional plot types. Three-dimensional plotting is one of the
functionalities that benefits immensely from viewing figures
interactively rather than statically in the notebook; recall that to use
interactive figures, you can use \%matplotlib notebook rather than
\%matplotlib inline when running this code.

    ** Line Plotting in 3D **

from sets of (x, y, z) triples. In analogy with the more common
two-dimensional plots discussed earlier, these can be created using the
ax.plot3D and ax.scatter3D functions. The call signature for these is
nearly identical to that of their two-dimensional counterparts, so you
can refer to Simple Line Plots and Simple Scatter Plots for more
information on controlling the output. Here we'll plot a trigonometric
spiral, along with some points drawn randomly near the line:

    \begin{Verbatim}[commandchars=\\\{\}]
{\color{incolor}In [{\color{incolor}4}]:} \PY{n}{plt}\PY{o}{.}\PY{n}{figure}\PY{p}{(}\PY{n}{figsize}\PY{o}{=}\PY{p}{(}\PY{l+m+mi}{10}\PY{p}{,}\PY{l+m+mi}{6}\PY{p}{)}\PY{p}{)}
        \PY{n}{ax} \PY{o}{=} \PY{n}{plt}\PY{o}{.}\PY{n}{axes}\PY{p}{(}\PY{n}{projection}\PY{o}{=}\PY{l+s+s1}{\PYZsq{}}\PY{l+s+s1}{3d}\PY{l+s+s1}{\PYZsq{}}\PY{p}{)}
        
        \PY{c+c1}{\PYZsh{} Data for a three\PYZhy{}dimensional line}
        \PY{n}{zline} \PY{o}{=} \PY{n}{np}\PY{o}{.}\PY{n}{linspace}\PY{p}{(}\PY{l+m+mi}{0}\PY{p}{,} \PY{l+m+mi}{15}\PY{p}{,} \PY{l+m+mi}{1000}\PY{p}{)}
        \PY{n}{xline} \PY{o}{=} \PY{n}{np}\PY{o}{.}\PY{n}{sin}\PY{p}{(}\PY{n}{zline}\PY{p}{)}
        \PY{n}{yline} \PY{o}{=} \PY{n}{np}\PY{o}{.}\PY{n}{cos}\PY{p}{(}\PY{n}{zline}\PY{p}{)}
        \PY{n}{ax}\PY{o}{.}\PY{n}{plot3D}\PY{p}{(}\PY{n}{xline}\PY{p}{,} \PY{n}{yline}\PY{p}{,} \PY{n}{zline}\PY{p}{,} \PY{l+s+s1}{\PYZsq{}}\PY{l+s+s1}{gray}\PY{l+s+s1}{\PYZsq{}}\PY{p}{)}
        
        \PY{c+c1}{\PYZsh{} Data for three\PYZhy{}dimensional scattered points}
        \PY{n}{zdata} \PY{o}{=} \PY{l+m+mi}{15} \PY{o}{*} \PY{n}{np}\PY{o}{.}\PY{n}{random}\PY{o}{.}\PY{n}{random}\PY{p}{(}\PY{l+m+mi}{100}\PY{p}{)}
        \PY{n}{xdata} \PY{o}{=} \PY{n}{np}\PY{o}{.}\PY{n}{sin}\PY{p}{(}\PY{n}{zdata}\PY{p}{)} \PY{o}{+} \PY{l+m+mf}{0.1} \PY{o}{*} \PY{n}{np}\PY{o}{.}\PY{n}{random}\PY{o}{.}\PY{n}{randn}\PY{p}{(}\PY{l+m+mi}{100}\PY{p}{)}
        \PY{n}{ydata} \PY{o}{=} \PY{n}{np}\PY{o}{.}\PY{n}{cos}\PY{p}{(}\PY{n}{zdata}\PY{p}{)} \PY{o}{+} \PY{l+m+mf}{0.1} \PY{o}{*} \PY{n}{np}\PY{o}{.}\PY{n}{random}\PY{o}{.}\PY{n}{randn}\PY{p}{(}\PY{l+m+mi}{100}\PY{p}{)}
        \PY{n}{ax}\PY{o}{.}\PY{n}{scatter3D}\PY{p}{(}\PY{n}{xdata}\PY{p}{,} \PY{n}{ydata}\PY{p}{,} \PY{n}{zdata}\PY{p}{,} \PY{n}{c}\PY{o}{=}\PY{n}{zdata}\PY{p}{,} \PY{n}{cmap}\PY{o}{=}\PY{l+s+s1}{\PYZsq{}}\PY{l+s+s1}{Greens}\PY{l+s+s1}{\PYZsq{}}\PY{p}{)}\PY{p}{;}
\end{Verbatim}


    \begin{center}
    \adjustimage{max size={0.9\linewidth}{0.9\paperheight}}{output_115_0.png}
    \end{center}
    { \hspace*{\fill} \\}
    
    Notice that by default, the scatter points have their transparency
adjusted to give a sense of depth on the page. While the
three-dimensional effect is sometimes difficult to see within a static
image, an interactive view can lead to some nice intuition about the
layout of the points. Use the \emph{scatter3D} or the \emph{plot3D}
method to plot a random walk in 3-dimensions in your exercise.

    ** Surface Plotting **

A surface plot is like a wireframe plot, but each face of the wireframe
is a filled polygon. Adding a colormap to the filled polygons can aid
perception of the topology of the surface being visualized:

    \begin{Verbatim}[commandchars=\\\{\}]
{\color{incolor}In [{\color{incolor}13}]:} \PY{n}{x} \PY{o}{=} \PY{n}{np}\PY{o}{.}\PY{n}{linspace}\PY{p}{(}\PY{o}{\PYZhy{}}\PY{l+m+mi}{6}\PY{p}{,} \PY{l+m+mi}{6}\PY{p}{,} \PY{l+m+mi}{50}\PY{p}{)}
         \PY{n}{y} \PY{o}{=} \PY{n}{np}\PY{o}{.}\PY{n}{linspace}\PY{p}{(}\PY{o}{\PYZhy{}}\PY{l+m+mi}{6}\PY{p}{,} \PY{l+m+mi}{6}\PY{p}{,} \PY{l+m+mi}{60}\PY{p}{)}
         
         \PY{n}{X}\PY{p}{,} \PY{n}{Y} \PY{o}{=} \PY{n}{np}\PY{o}{.}\PY{n}{meshgrid}\PY{p}{(}\PY{n}{x}\PY{p}{,} \PY{n}{y}\PY{p}{)}
         \PY{n}{Z}\PY{o}{=}\PY{n}{np}\PY{o}{.}\PY{n}{sin}\PY{p}{(}\PY{n}{X}\PY{p}{)}\PY{o}{*}\PY{n}{np}\PY{o}{.}\PY{n}{sin}\PY{p}{(}\PY{n}{Y}\PY{p}{)}
\end{Verbatim}


    \begin{Verbatim}[commandchars=\\\{\}]
{\color{incolor}In [{\color{incolor}18}]:} \PY{n}{np}\PY{o}{.}\PY{n}{shape}\PY{p}{(}\PY{n}{Z}\PY{p}{)}
\end{Verbatim}


\begin{Verbatim}[commandchars=\\\{\}]
{\color{outcolor}Out[{\color{outcolor}18}]:} (60, 50)
\end{Verbatim}
            
    \begin{Verbatim}[commandchars=\\\{\}]
{\color{incolor}In [{\color{incolor}8}]:} \PY{n}{plt}\PY{o}{.}\PY{n}{figure}\PY{p}{(}\PY{n}{figsize}\PY{o}{=}\PY{p}{(}\PY{l+m+mi}{10}\PY{p}{,}\PY{l+m+mi}{6}\PY{p}{)}\PY{p}{)}
        \PY{n}{ax} \PY{o}{=} \PY{n}{plt}\PY{o}{.}\PY{n}{axes}\PY{p}{(}\PY{n}{projection}\PY{o}{=}\PY{l+s+s1}{\PYZsq{}}\PY{l+s+s1}{3d}\PY{l+s+s1}{\PYZsq{}}\PY{p}{)}
        \PY{n}{ax}\PY{o}{.}\PY{n}{plot\PYZus{}surface}\PY{p}{(}\PY{n}{X}\PY{p}{,} \PY{n}{Y}\PY{p}{,} \PY{n}{Z}\PY{p}{,} \PY{n}{rstride}\PY{o}{=}\PY{l+m+mi}{1}\PY{p}{,} \PY{n}{cstride}\PY{o}{=}\PY{l+m+mi}{1}\PY{p}{,}
                        \PY{n}{cmap}\PY{o}{=}\PY{l+s+s1}{\PYZsq{}}\PY{l+s+s1}{viridis}\PY{l+s+s1}{\PYZsq{}}\PY{p}{,} \PY{n}{edgecolor}\PY{o}{=}\PY{l+s+s1}{\PYZsq{}}\PY{l+s+s1}{none}\PY{l+s+s1}{\PYZsq{}}\PY{p}{)}
        \PY{n}{ax}\PY{o}{.}\PY{n}{set\PYZus{}title}\PY{p}{(}\PY{l+s+s1}{\PYZsq{}}\PY{l+s+s1}{surface}\PY{l+s+s1}{\PYZsq{}}\PY{p}{)}\PY{p}{;}
\end{Verbatim}


    \begin{center}
    \adjustimage{max size={0.9\linewidth}{0.9\paperheight}}{output_120_0.png}
    \end{center}
    { \hspace*{\fill} \\}
    
    \hypertarget{animations}{%
\subsection{Animations}\label{animations}}

To display function animations, matplotlib has a special module to carry
out animations. The \emph{FuncAnimation} command takes several
arguments.

\begin{verbatim}
anim = animation.FuncAnimation(fig, animate, init_func=init, frames=200, interval=20, blit=False)
\end{verbatim}

\begin{itemize}
\tightlist
\item
  \textbf{fig} figure to draw the data in
\item
  \textbf{animate} the function which varies to recalculate the plotting
  data, it gets the frame \(i\) number as an argument
\item
  \textbf{init\_func} function which initialzes the line data arrays
\item
  \textbf{frames} number of frames which shall be displayed
\item
  \textbf{intervall} time intervall for the display in ms
\item
  \textbf{blit} this is a special way of buffering, doen't work on osx,
  but might on Windows
\end{itemize}

    \begin{Verbatim}[commandchars=\\\{\}]
{\color{incolor}In [{\color{incolor}502}]:} \PY{o}{\PYZpc{}}\PY{k}{matplotlib}
\end{Verbatim}


    \begin{Verbatim}[commandchars=\\\{\}]
Using matplotlib backend: MacOSX

    \end{Verbatim}

    \begin{Verbatim}[commandchars=\\\{\}]
{\color{incolor}In [{\color{incolor}28}]:} \PY{k+kn}{import} \PY{n+nn}{numpy} \PY{k}{as} \PY{n+nn}{np}
         \PY{k+kn}{from} \PY{n+nn}{matplotlib} \PY{k}{import} \PY{n}{pyplot} \PY{k}{as} \PY{n}{plt}
         \PY{k+kn}{from} \PY{n+nn}{matplotlib} \PY{k}{import} \PY{n}{animation}
         
         \PY{c+c1}{\PYZsh{} First set up the figure, the axis, and the plot element we want to animate}
         \PY{n}{fig} \PY{o}{=} \PY{n}{plt}\PY{o}{.}\PY{n}{figure}\PY{p}{(}\PY{p}{)}
         \PY{n}{ax} \PY{o}{=} \PY{n}{plt}\PY{o}{.}\PY{n}{axes}\PY{p}{(}\PY{n}{xlim}\PY{o}{=}\PY{p}{(}\PY{l+m+mi}{0}\PY{p}{,} \PY{l+m+mi}{2}\PY{p}{)}\PY{p}{,} \PY{n}{ylim}\PY{o}{=}\PY{p}{(}\PY{o}{\PYZhy{}}\PY{l+m+mi}{2}\PY{p}{,} \PY{l+m+mi}{2}\PY{p}{)}\PY{p}{)}
         \PY{n}{line}\PY{p}{,} \PY{o}{=} \PY{n}{ax}\PY{o}{.}\PY{n}{plot}\PY{p}{(}\PY{p}{[}\PY{p}{]}\PY{p}{,} \PY{p}{[}\PY{p}{]}\PY{p}{,} \PY{n}{lw}\PY{o}{=}\PY{l+m+mi}{2}\PY{p}{)}
         
         \PY{c+c1}{\PYZsh{} initialization function: plot the background of each frame}
         \PY{k}{def} \PY{n+nf}{init}\PY{p}{(}\PY{p}{)}\PY{p}{:}
             \PY{n}{line}\PY{o}{.}\PY{n}{set\PYZus{}data}\PY{p}{(}\PY{p}{[}\PY{p}{]}\PY{p}{,} \PY{p}{[}\PY{p}{]}\PY{p}{)}
             \PY{k}{return} \PY{n}{line}\PY{p}{,}
         
         \PY{c+c1}{\PYZsh{} animation function.  This is called sequentially}
         \PY{k}{def} \PY{n+nf}{animate}\PY{p}{(}\PY{n}{i}\PY{p}{)}\PY{p}{:}
             \PY{n}{x} \PY{o}{=} \PY{n}{np}\PY{o}{.}\PY{n}{linspace}\PY{p}{(}\PY{l+m+mi}{0}\PY{p}{,} \PY{l+m+mi}{2}\PY{p}{,} \PY{l+m+mi}{1000}\PY{p}{)}
             \PY{n}{y} \PY{o}{=} \PY{n}{np}\PY{o}{.}\PY{n}{sin}\PY{p}{(}\PY{l+m+mi}{2} \PY{o}{*} \PY{n}{np}\PY{o}{.}\PY{n}{pi} \PY{o}{*} \PY{p}{(}\PY{n}{x} \PY{o}{\PYZhy{}} \PY{l+m+mf}{0.01} \PY{o}{*} \PY{n}{i}\PY{p}{)}\PY{p}{)}
             \PY{n}{line}\PY{o}{.}\PY{n}{set\PYZus{}data}\PY{p}{(}\PY{n}{x}\PY{p}{,} \PY{n}{y}\PY{p}{)}
             \PY{k}{return} \PY{n}{line}\PY{p}{,}
         
         \PY{n}{anim} \PY{o}{=} \PY{n}{animation}\PY{o}{.}\PY{n}{FuncAnimation}\PY{p}{(}\PY{n}{fig}\PY{p}{,} \PY{n}{animate}\PY{p}{,} \PY{n}{init\PYZus{}func}\PY{o}{=}\PY{n}{init}\PY{p}{,}
                                        \PY{n}{frames}\PY{o}{=}\PY{l+m+mi}{200}\PY{p}{,} \PY{n}{interval}\PY{o}{=}\PY{l+m+mi}{20}\PY{p}{,} \PY{n}{blit}\PY{o}{=}\PY{k+kc}{True}\PY{p}{)}
\end{Verbatim}


    \begin{center}
    \adjustimage{max size={0.9\linewidth}{0.9\paperheight}}{output_123_0.png}
    \end{center}
    { \hspace*{\fill} \\}
    
    There are a number of other options for creating animated graphics in
python. You will learn about pygame for example in the exercises.

    \begin{center}\rule{0.5\linewidth}{\linethickness}\end{center}

\hypertarget{whats-next}{%
\section{What's next}\label{whats-next}}

    After you have carried out for yourself all the exercises before, you
are fit in plotting in various ways.


    % Add a bibliography block to the postdoc
    
    
    
    \end{document}
